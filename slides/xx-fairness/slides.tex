\documentclass[11pt,compress,t,notes=noshow, xcolor=table]{beamer}
\input{../../style/preamble}
\input{../../latex-math/basic-math}
\input{../../latex-math/basic-ml}

\newcommand{\dv}{\mathbf{d}} % vector d (bold)
\newcommand{\av}{\mathbf{a}} % vector a (bold)
\newcommand{\Dspace}{\delta} % Decision space d
\newcommand{\Aspace}{\mathcal{A}} % Protected attribute space A

\usepackage{multicol}


\title{Advanced Machine Learning}
\date{}

\begin{document}

\lecturechapter{Algorithmic Fairness}
\lecture{Advanced Machine Learning}


%% Google Slides source of figures: 
%%  https://docs.google.com/presentation/d/14Vse6gkQ6PaVPsCr_rQXXBlN21Yu8tVDmU7v-YLoqQQ/edit?usp=sharing

\begin{vbframe}{ALGORITHMIC FAIRNESS}
 \begin{center}
 \vspace{-0.5cm}
 \includegraphics[width=0.62\textwidth]{figures/scale.png}
 \end{center}

 \begin{itemize}
  \item Machine learning (ML) based systems increasingly permeate society
  \item Models can replicate existing injustices or introduce new ones
  \item Automated decisions can disproportionately harm vulnerable individuals
 \end{itemize}
\end{vbframe}


\begin{vbframe}{ALGORITHMIC FAIRNESS}
\begin{columns}
\begin{column}{0.5\textwidth}
\begin{center}
 \vspace{-0.5cm}

 \textbf{Medicine}

 \includegraphics[height=0.2\textheight]{figures/gender_imbalance.png}
 \tiny{www.pnas.org/content/117/23/12592}

 \vspace{1.5cm}

 \textbf{Hiring}

 \includegraphics[height=0.2\textheight]{figures/hiring.png}
 \tiny{https://interaktiv.br.de/ki-bewerbung/en/}

\end{center}
\end{column}
\begin{column}{0.5\textwidth}
\begin{center}
 \vspace{-0.5cm}

 \textbf{Criminal Justice}

 \includegraphics[height=0.2\textheight]{figures/compas.png}
 \tiny{{https://www.propublica.org/article/machine-bias-risk-assessments-in-criminal-sentencing}}

 \vspace{1.5cm}

 \textbf{Search Results}

 \includegraphics[height=0.2\textheight]{figures/black_girls.png}
 \tiny{https://time.com/5209144/google-search-engine-algorithm-bias-racism/}
\end{center}
\end{column}
\end{columns}
\end{vbframe}


% %% Possibly skip this slide
% \begin{vbframe}{What is Bias?}
% \begin{center}
% \vspace{1.5cm}
% \enquote{Bias as a systematic error or an unexpected tendency to favor one outcome over another}
% (Mehrabi et al., 2019)
% \vspace{1.5cm}
% \enquote{Fairness is the “absence of any prejudice or favoritism towards an individual or a group based on their intrinsic or acquired traits in the context of decision-making}
% (Mehrabi et al., 2019)
% \end{center}
% \end{vbframe}


\begin{vbframe}{Sources of Bias}
\begin{center}
\includegraphics[scale=1]{figures/bias_overview.pdf}
\end{center}
\tiny{Adapted from S. Mitchell et al., Algorithmic fairness: Choices, assumptions, and definitions, 2021}
\end{vbframe}

\begin{vbframe}{Historical Bias}
  \begin{itemize}
  \item Historical data often contains biases, e.g. under-representation of minority groups
  \item Models can pick up existing biases
  \item As a result, biases are perpetuated into the future
  \end{itemize}
  \vspace{1cm}
\begin{center}
\includegraphics[width=0.8\textwidth]{figures/historical_bias.png}

\tiny{Twitter: \@math\_rachel 01.05.2019}
\end{center}
\end{vbframe}

\begin{vbframe}{Representation Bias}
 \vspace{-0.25cm}
  \begin{itemize}
  \item Over- or under-representation of specific sub-population can lead to models that only predict well for majority groups
  \item Models need to be evaluated across a representative sample of the target population
  \item Example: We can only know if a person paid back a loan if we gave out a loan in the first place
  \end{itemize}
\begin{center}
\includegraphics[width=0.6\textwidth]{figures/gendershades.jpeg}

\tiny{gendershades.org}
\end{center}
\end{vbframe}

\begin{vbframe}{Other Sources of Bias}

  \begin{itemize}
  \item \textbf{Measurement Bias} Difference in how a given variable is measured in different sub-populations
    \begin{itemize}
      \begin{footnotesize}
      \item Increased policing in some post codes lead to more prior arrests
      \item Better data quality between different hospitals
      \end{footnotesize}
    \end{itemize}
  \item \textbf{Model Bias} Biases introduced during modelling, e.g. due to under-specified models
      \begin{itemize}
      \begin{footnotesize}
      \item Models make more errors for darker skin tones due to insufficient data
      \item Models pick up spurious correlations in the data
      \end{footnotesize}
      \end{itemize}
  \item \textbf{Feedback Loops} Model decisions shape data collected in the future
      \begin{itemize}
      \begin{footnotesize}
      \item Lead to representation bias if e.g. sub-populations are systematically excluded
      \item People and ML systems 'pick up' miss-representation from search engines. 
      \end{footnotesize}
\end{itemize}
\end{itemize}

\vfill
\tiny{Mehrabi et al., A Survey on Bias and Fairness in Machine Learning, 2020}
\end{vbframe}

\begin{vbframe}{Types of Harms}
  If not accounted for, \textit{biases} can lead to several \textbf{harms}
  \begin{itemize}
  \begin{small}
  \item \textbf{Allocation}: A ressource is allocated unevenly across individuals
  \item \textbf{Quality-of-service}: Systems fail disproportionately for certain groups of individuals.
  \item \textbf{Stereotyping}: Systems re-inforce existing stereotypes
  \item \textbf{Denigration}: Systems are offensive towards individuals
  \item \textbf{Representation}: Under- or overrepresentation of certain groups
  \end{small}
  \end{itemize}

\begin{columns} 
\begin{column}{0.5\textwidth}
\begin{center}
\vfill
 \includegraphics[height=0.25\textheight]{figures/gorilla.png}
 \tiny{Twitter: \@jackyalcine 29.06.2015}
 \vfill
\end{center}
\end{column}
\begin{column}{0.5\textwidth}
\begin{center}
 \vspace{-0.5cm}
 \includegraphics[height=0.3\textheight]{figures/doctor.png}
 \tiny{google.com search for doctor (May, 2021)}
\end{center}
\end{column}
\end{columns}
\vfill
\tiny{H. Weerts, An introduction to algorithmic fairness, 2021}
\end{vbframe}



\begin{vbframe}{Auditing Models for Potential Harms}
  Models Biases in data can have \textbf{ethical}, \textbf{legal} and \textbf{regulatory} consequences \\
  For a more formal treatment, we introduce additional notation:
  \begin{itemize}
    \item \textbf{Protected attribute} Fairness definitions often require a protected \emph{class} or attribute w.r.t which models should be fair.
    We denote this protected attribute $A$ with $\av$. For simplicity, we assume that $\av^{(i)} \in \Aspace = \{0,1\}$ is a binary variable
    \item \textbf{Decision space} In order to make the difference between a model's prediction $\fxh$ and a decision derived from this prediction explicit,
    we denote the decision  with $\dv$. For simplicity, we again assume binary decisions $\dv^{(i)} \in \Dspace = \{0, 1\}$
  \end{itemize}

This notation can be extended to allow for multi-class or regression outcomes as well as more complex protected attributes,
that e.g. allow accounting for non-binary protected classes or \emph{intersectional notions}, e.g. race $\land$ gender.
\end{vbframe}

\begin{vbframe}{Mathematical Notions of Bias - Overview}
\begin{center}
  \vspace{.5cm}
  \includegraphics[width=\textwidth]{figures/fairness_definitions.png}
\end{center}
\end{vbframe}

\begin{vbframe}{No fairness through unawareness}
  A naive proposal to reduce harms from ML models is to simply remove the protected attribute.
  \textbf{But:} It's not that simple - models can pick up the information through other variables!
  \vspace{-.4cm}
  \begin{columns} 
    \begin{column}{0.5\textwidth}
    \begin{center}
    \textbf{Direct Discrimination}
    \vspace{.2cm}
    \begin{tikzpicture}
      % x node set with absolute coordinates
      \node[state] (a) at (0,0) {$Race$};
      \node[state] (x1) [right =of a] {$ZIP$};
      \node[state] (x2) [right =of x1] {$Income$};
      \node[state] (y) [below =of x1] {$Credit$};
      % Directed edge
      \draw [->, red] (a) -- (y);
      \draw [->] (x1) -- (y);
      \draw [->] (x2) -- (y);
    \end{tikzpicture}
    \end{center}
    \vfill
    $\rightarrow$ The model directly uses race as a feature.
    \end{column}
    \begin{column}{0.5\textwidth}
      \begin{center}
        \textbf{Indirect Discrimination}
        \vspace{.2cm}
        \begin{tikzpicture}
          % x node set with absolute coordinates
          \node[state] (a) at (0,0) {$Race$};
          \node[state] (x1) [right =of a] {$ZIP$};
          \node[state] (x2) [right =of x1] {$Income$};
          \node[state] (y) [below =of x1] {$Credit$};
          % Directed edge
          \draw [->, red] (a) -- (x1);
          \draw [->] (x1) -- (y);
          \draw [->] (x2) -- (y);
        \end{tikzpicture}
        \end{center}
    \vfill
    $\rightarrow$ The model picks up information about the race through the proxy-variable ZIP-code.
    \end{column}
    \end{columns}
  \vfill
\end{vbframe}

\begin{vbframe}{Group fairness definitions}
  Several fairness definitions based on differences between protected groups have been proposed.
  \begin{itemize}
    \item \textbf{Statistical Parity}: The chance to get the favourable outcome is equal across two groups. This is also 
          called \textit{demographic parity}.
          \[
          P(\hat{Y} = 1 | A = 0) = P(\hat{Y} = 1 | A = 1)   
          \]
    \item \textbf{Equalized Opportunity}: The chance to \emph{correctly} be assigned the favourable outcome is independent of 
          the protected attribute. 
          \[
            P(\hat{Y} = 1 | A = 0, Y = 1) = P(\hat{Y} = 1 | A = 1, Y = 1)   
          \]

    \item \textbf{Accuracy Parity}: The accuracy is equal in both groups.
    
    \[ \begin{aligned}
    & P(\hat{Y} = 1 | A = 0, Y = 1) + P(\hat{Y} = 0 | A = 0, Y = 0) = \\
    & P(\hat{Y} = 1 | A = 1, Y = 1) + P(\hat{Y} = 0 | A = 1, Y = 0)
    \end{aligned} \]
  \end{itemize}
\end{vbframe}

\begin{vbframe}{Perspective: Based on predicted outcome}
    \textbf{}
    \begin{itemize}
      \item Statistical parity requires equality in the predicted outcome.\\
      E.g. hire candidates \textbf{independent} of qualification.
      \item If the underlying qualifications are not distributed equally across groups, we need to sacrifice \emph{utility} to achieve statistical parity.
    \end{itemize}
    \begin{center}
      \includegraphics[]{figures/statistical_parity}
    \end{center}
    $\rightarrow$ Enforcing equal positive rates might require hiring unqualified candidates.\\
    \textbf{Danger:} If the bias comes from the real world (e.g. societal bias), enforcing statistical parity can also lead to adverse effects in the long term.
\end{vbframe}

\begin{vbframe}{Perspective: Based on true \& predicted outcome}
  \textbf{}
  \begin{itemize}
    \item Other fairness notions require equality of some error notions, e.g. false positive rates.\\
    E.g. hire \emph{qualified} candidates at equal rates across groups.
    \item Error based notions are often more intuitive and easy to communicate.
    \item Can help to idenitify representation or model bias
    \item Error based notions do not account for systemic injustices in the world -- if e.g. labels are biased, we can still be \emph{fair} according to error-based notions. 
  \end{itemize}
\end{vbframe}

\begin{vbframe}{Reminder: Confusion Matrix}
  The confusion matrix is a $2 \times 2$ contingency table of predictions $\hat{y}$ and true labels $y$.
  Several evaluation metrics can be derived from a confusion matrix:
  \vfill
  \begin{center}
  \includegraphics[width=0.8\textwidth]{figures/roc-confusion_matrix.png}
  \end{center}
  $\rightarrow$ Many fairness metrics can be expressed as entries of the confusion matrix
\end{vbframe}
   
\begin{vbframe}{Fairness Tensor}
  We can represent labels \& predictions as a \emph{fairness tensor} (Kim et al., 2020).
  Fairness tensors are 3-dimensional, stacked confusion matrices:

  \[
  Z = \left[
    \begin{bmatrix}
    TP_1  & FP_1 \\
    FN_1  & TN_1
    \end{bmatrix} ^{A=1}
    ,
    \begin{bmatrix}
      TP_0  & FP_0 \\
      FN_0  & TN_0
    \end{bmatrix} ^{A=0} 
    \right]
  \]
  \vspace{.2cm}
  For $z = (TP_1,FN_1,FP_1,TN_1,TP_0,FN_0,FP_0,TN_0)^T / N$, we can express a large variety of fairness metrics as 
  linear $\phi(x) = A \cdot z$ or quadratic functions $\phi(x) = z^T \cdot B \cdot z$ by choosing an appropriate matrix $A$ or $B$. \\
  \vspace{.2cm}
  \textbf{Example:} \\ We choose $A = (N_1, 0, N_1, 0, N_0, 0, N_0, 0) / N$, where $N_{a}$ is the sum of entries in the confusion matrix for protected grou $a$.
  We can now express \textbf{statistical parity} as $A \cdot z = 0$.
\end{vbframe}
\begin{vbframe}{Incompatibility of Fairness Metrics}

  Some fairness metrics can not be jointly satisfied, except for trivial settings.
  As a result, we can not build a universal fairness metric that satisfies all fairness requirements.
  This has been shown for several combinations of fairness metrics, e.g. equal true positive, false positive and false negative rates.\\
  \vspace{.2cm}
  We can formally show this using the fairness tensor $z$ and matrices $A_{TPR}, A_{FPR}, A_{FNR}$ encoding the fairness metrics.
  Showing that fairness metrics are compatible now requires showing that the system of equations 
  \[ \begin{bmatrix} A_{TPR} \\ A_{FPR} \\ A_{FNR} \end{bmatrix} \cdot z  = \begin{bmatrix} 0 \\ 0 \\ 0 \end{bmatrix} \]
  has a solution. If no solution $z$ exists, the metrics are incompatible.

\end{vbframe}

\begin{vbframe}{Fairness Metrics - Closing Thoughts}
  \begin{itemize}
    \item Statistical group fairnes metrics require translating ethical considerations of what is \emph{fair} into mathematical formulas.
    \item To draw meaningful conclusions, we need to evaluate fairness metrics on a \textbf{representative} data set. 
    \item Fairness metrics reduce a wide variety of important considerations into a single number -- they are not designed to guarantee that a system is fair.
    \item Incompatibility between fairness metrics implies that we might need trade-offs between fairness metrics.
  \end{itemize}
\end{vbframe}

\begin{vbframe}{Preventing \& Mitigating Harms - Documentation}
    Harm from machine learning models can often be prevented by improving documentation of existing datasets \& models.
    To provide an example, usage of datasets or models outside of their intended use can often lead to harm, even if the models are carefully validated.
    \begin{itemize}
      \item \textbf{Dataset documentation} Includes information on the dataset, sampling mechanisms and intended use.
      Can inform model developers on possible representation biases or other problems with the data. \\
      \textbf{Example:} Datasheets for datasets
      \item \textbf{Model documentation} Includes information about the model, used data and hyperparameters. 
      Can inform users on the relevant performance metrics, intended use of a given model. \\
      \textbf{Example:} Modelcards for ML models
      \item \textbf{Fairness reports} Include information about performed fairness audits.
    \end{itemize}
\end{vbframe}

\begin{vbframe}{Preventing \& Mitigating Harms - Bias Mitigation}
  Several \emph{bias mitigation techniques} have been proposed:
  \begin{itemize}
    \item \textbf{Pre-processing}: Transform data to make subsequently trained models fairer.
    \item \textbf{In-processing}: Learn a model that directly incorporates fairness constraints.
    \item \textbf{Post-processing}: Adapt model predictions to satisfy fairness constraints.
  \end{itemize}
  \textbf{Example:}\\
  Re-weighing (Kamiran, 2012) proposes to use sample weights that are inverse to the frequency of labels and predictions in the data.
\end{vbframe}

\begin{vbframe}{Preventing \& Mitigating Harms - Recourse}
  Fair treatment of individuals subject to a decision making systems decisions can often not only be achieved solely through algorithmic means
  but requires recourse, accountability \& interpretability.
  \begin{itemize}
    \item \textbf{Accountability}: Automated systems will make errors - developers need to ensure that humans responsible for addressing such errors exist and have the means to address such errors.
    \item \textbf{Interpretability}: Interpretability techniques can help to identify possible problems in the data or the model, e.g. spurious correlations picked up by the model.
    \item \textbf{Recourse}: Individuals subject to automated decisions should have access to an explanation on how the decision was made and what steps can be taken to address unfavourable predictions.
  \end{itemize} 
\end{vbframe}


\begin{vbframe}{Further considerations}
  \begin{itemize}
    \item \textbf{Intersectionality}: Fairness considerations should often hold across intersectional groups, e.g. $race \land gender$.
    \item \textbf{Intervention design}: Instead of ensuring a given intervention is fair, it can often be helpful to consider the intervention we wish to deploy. \\
    \textbf{Example:} Instead of penalizing defendants for not showing up to court, provide them with means of transportation.
    \item \textbf{Stakeholder participation:} Developing ML models should take the perspective of all stakeholders such as the individuals affected by the intervention and advocacy groups.
    \item \textbf{Long-term perspective:} Existing metrics only consider the short-term and do not take its long-term impact into account. This might lead to adverse effects in the long-term.
  \end{itemize}
\end{vbframe}

\begin{vbframe}{Resources}
\begin{itemize}
\item Fairness and Machine Learning - Limitations and Opportunities, Barocas et al., 2019
\item Algorithmic Fairness: Choices, Assumptions, and Definitions, Mitchell et al., 2021
\item A Survey on Bias and Fairness in Machine Learning, Mehrabi et al., 2020
\item An Introduction to Algorithmic Fairness, H.J.P Weerts, 2021
\item FACT: A Diagnostic for Group Fairness Trade-offs, Kim et al., 2020
\item Data preprocessing techniques for classification without discrimination, Kamiran et al., 2012
\item Fairness Through Awareness, Dwork et al., 2012
\end{itemize}
\end{vbframe}

\begin{vbframe}{Software}
  \begin{itemize}
  \item \texttt{fairlearn} (Python)
  \item \texttt{aif360} (Python)
  \item \texttt{fairmodels} (R)
  \item \texttt{mlr3fairness} (R)
  \end{itemize}
  \end{vbframe}

\endlecture
\end{document}