\documentclass[11pt,compress,t,notes=noshow, xcolor=table]{beamer}
\input{../../style/preamble}
\input{../../latex-math/basic-math}
\input{../../latex-math/basic-ml}
\input{../../latex-math/ml-online}

\usepackage{multicol}

\newcommand{\titlefigure}{figure/FTL_illustration}
\newcommand{\learninggoals}{
	\item Prove that FTL works for online quadratic optimization problems
	%  \item Get to know the FTRL algorithm as a stable alternative
}

\title{Advanced Machine Learning}
\date{}

\begin{document}
	
	\lecturechapter{Follow the leader for OQO problems}
	\lecture{Advanced Machine Learning}
	
	
	
	\sloppy




\begin{frame} 
\frametitle{FTL for OQO problems}
%	
\footnotesize
%
\begin{itemize}
	%		
	\item One popular instantiation of the online learning problem is the problem of \emph{online quadratic optimization} (OQO).
	%		
	\item In its most general form, the loss function is thereby defined as 
	%	
	\begin{align*}
		%		 \label{defi_loss_quadratic_optim}
		%		
		\l(a_t,z_t) =  \frac12 \norm{a_t-z_t}^2,
		%		
	\end{align*}
	%	
	where $\Aspace,\Zspace \subset \R^d.$ 
	%		
	%		
	\item  {\visible<1->{ \textbf{Proposition:}
			%	
			Using FTL on any online quadratic optimization problem with $\Aspace = \mathbb{R}^d$ and $V = \sup\limits_{z \in \Zspace} \norm{z}$, leads to a regret of  
			%	 
			\begin{equation*}
				%	
				R_T^{\FTL}  \leq 4V^2 \, (\log(T) + 1 ).
				%		
	\end{equation*} }}
	%
	\item []
	%	
	
\end{itemize}
%	
\end{frame}


\begin{frame} 
	\frametitle{FTL for OQO problems: Analysis}
	%	
	\small
	\begin{itemize}		
		
		\item \textbf{Proof:}
		%	
		\begin{itemize}
				%				
				\item In the following, we denote  $a_1^{\FTL}, a_2^{\FTL}, \ldots$  simply by  $a_1, a_2, \ldots$  	
				%		
				\pause
				\fbox{\begin{minipage}{0.9\textwidth}
						\begin{align*}
					%	
						&\mbox{\textbf{Reminder (Useful Lemma):} \qquad } \\
						&R_T^{\FTL} {\color{black} \leq}    {\color{black} \sum\nolimits_{t=1}^T \l(a_t^{\FTL},z_t) }-  {\color{black}\sum\nolimits_{t=1}^T  \l(a_{t+1}^{\FTL},z_t)}
					%	
				\end{align*}
				\end{minipage}
				}
				
				\pause	
				 \item  Using this lemma, we just have to show that
				%
				\begin{align} \label{ineq_help_FTL_quadr}
						%	
						\sum\limits_{t=1}^T \left(\l(a_t,z_t) - \l(a_{t+1},z_t)  \right) \leq 4L^2 \cdot \left(\log(T) + 1 \right).
						%	
					\end{align}
				%						
%				since $R_T^{\FTL} \stackrel{{\color{blue} \mbox{Lemma}}}{\leq} \sum\nolimits_{t=1}^T \left(\l(a_t,z_t) - \l(a_{t+1},z_t)  \right).$ 
				%	
				\pause			
				 \item So, we will prove (\ref{ineq_help_FTL_quadr}).
				%				
				For this purpose, we compute the explicit form of the actions of FTL for this type of online learning problem.
				%				
			\end{itemize}
		%		
	\end{itemize}
\end{frame}
%
%
\begin{frame} 
	\frametitle{FTL for OQO problems: Analysis}
	%	
	\small
	\begin{itemize}	
		%	
		\item[]
		\begin{itemize}	 \small
				%				
				\item Claim: It holds that $a_t = \frac{1}{t - 1} \cdot \sum\nolimits_{s = 1}^{t-1} z_s,$ if $\l(a,z)=\frac12\norm{a-z}^2 .$ 
				%			
				\pause	
				\begin{itemize} \small
						%				
						 \item	Recall that $$a_t^{\FTL} = \argmin_{a \in \mathcal{A}} \sum_{s=1}^{t-1} \l(a,z_s)  = \argmin_{a \in \mathcal{A}} \sum_{s=1}^{t-1} \frac12 \norm{a - z_s}^2. $$
						%		
						\pause				
						 \item So, we have to find the minimizer of the function $$f(a):= \sum_{s=1}^{t-1} \frac12 \norm{a - z_s}^2  =   \sum_{s=1}^{t-1} \frac12 (a - z_s)^\top (a-z_s).$$
						%		
						\pause				
						 \item Compute  $\nabla f(a) = \sum_{s=1}^{t-1}  a  - z_s =  (t-1) a  - \sum_{s=1}^{t-1} z_s,$ which we set to zero and solve with respect to $a$ to obtain the claim.\\
						%
						{\tiny ($f$ is convex, so that this leads indeed to a minimizer.)}
						%	
						%				
					\end{itemize} 
				%				
			\end{itemize}
		%		
	\end{itemize}
\end{frame}
%
%
\begin{frame} 
	\frametitle{FTL for OQO problems: Analysis}
	%	
	\small
	\begin{itemize}	
		%	
		\item[]
		\begin{itemize}	 
				%
				\item Hence, $a_t$ is the empirical average of $z_1, \ldots, z_{t-1}$ and we can provide the following incremental update formula for its computation
				%
				\begin{equation*}
						%
						\begin{split}
								%	
								a_{t+1} = \tfrac{1}{t} \cdot \sum\limits_{s = 1}^{t} z_s
								%	
								&= \tfrac{1}{t}\left(z_t + \sum\limits_{s = 1}^{t-1} z_s \right)  \\
%								
								&= \tfrac{1}{t}(z_t + (t-1) a_t) 
								%	
								 =  \tfrac{1}{t} z_t + \left(1 - \tfrac{1}{t} \right)  a_t.
								%	
							\end{split}
						%
					\end{equation*} 
				%
				%
				\pause
				 \item From the last display we derive that
				%
				\begin{equation*}
						%	
						\begin{split}
								% 
								a_{t+1} - z_t = \left(1 - \tfrac{1}{t} \right) \cdot a_t + \tfrac{1}{t} z_t - z_t = \left(1 - \tfrac{1}{t} \right) \cdot (a_t - z_t).
								%
							\end{split}
						%	
					\end{equation*}
				%	
				\pause		
				 \item Claim:
				%
				\begin{align*} 
						%				\label{help_ineq_FTL_analysis}
						%	
						\l(a_t,z_t) - \l(a_{t+1},z_t) \leq \tfrac{1}{t} \cdot \norm{a_t - z_t}^2. \tag{2}
						%	
					\end{align*}
				%	
			\end{itemize}
		%
	\end{itemize}
\end{frame}


\begin{frame} 
	\frametitle{FTL for OQO problems: Analysis}
	%	
	
	\fbox{\begin{minipage}{0.99\textwidth}
					\begin{equation*}
		%	
		\begin{split}
			% 
			\mbox{\textbf{Reminder:} \quad }	a_{t+1} - z_t = \left(1 - \tfrac{1}{t} \right) \cdot (a_t - z_t).
			%
		\end{split}
		%	
	\end{equation*}
\end{minipage}
}
	
	\small
	\begin{itemize}	
		%	
		\item[]
		\begin{itemize}	 
				%
				\item Indeed, this can be seen as follows
				%
				{\footnotesize
						\begin{align*}
								%
								\l(a_t,z_t) - \l(a_{t+1},z_t) 
								%
								&= \tfrac{1}{2}\norm{a_t - z_t}^2 - \frac{1}{2}\norm{a_{t+1} - z_{t}}^2 \\
								%
								 &= \tfrac{1}{2} \left( \norm{a_t - z_t}^2 - \norm{a_{t+1} - z_{t}}^2 \right) \\
								%
								%				\Big[\mbox{Using \ }a_{t+1} - z_t  = \left(1 - \tfrac{1}{t} \right) \cdot (a_t - z_t) \Big] \quad  
								 &= \tfrac{1}{2} \left( \norm{a_t - z_t}^2  - \norm{\left(1 - \tfrac{1}{t} \right) \cdot (a_t - z_t)}^2 \right).
								%
								%
							\end{align*}	
					}
				\pause 
				\item And from this,
								{\footnotesize
					\begin{align*}
						%
						\l(a_t,z_t) - \l(a_{t+1},z_t) 
						%
						%
						&= \tfrac{1}{2} \left( \norm{a_t - z_t}^2  - \left(1 - \tfrac{1}{t} \right)^2 \cdot \norm{a_t - z_t}^2 \right) \\
						%
						&= \tfrac{1}{2} \left(1 - \left(1 - \tfrac{1}{t} \right)^2 \right) \cdot \norm{a_t - z_t}^2 \\
						%
						&= \left(\tfrac{1}{t} - \tfrac{1}{2t^2}\right) \cdot \norm{a_t - z_t}^2 \\
						%
						&\leq \tfrac{1}{t} \cdot \norm{a_t - z_t}^2.
						%
					\end{align*}	
				}
				
				%
			\end{itemize}
	\end{itemize}
\end{frame}


\begin{frame} 
	\frametitle{FTL for OQO problems: Analysis}
	%	
	\small
	


	\begin{itemize}	
		%	
		


		\pause
		\item[] 
		
			\fbox{\begin{minipage}{0.9\textwidth}
				\begin{align}
					%				 \label{help_ineq_FTL_analysis}
					%	
					\mbox{\textbf{Reminder:} \quad }\l(a_t,z_t) - \l(a_{t+1},z_t) \leq \tfrac{1}{t} \cdot \norm{a_t - z_t}^2 \tag{2}.
					%	
				\end{align}	 
			\end{minipage}
		}
		
		\begin{itemize}	
				\footnotesize
				
				\item Since by assumption $L = \sup\limits_{z \in \Zspace} \norm{z}$ and $a_t$ is the empirical average of $z_1, \ldots, z_{t-1}$, we have that $\norm{a_t} \leq L.$
				%
				\pause
				 \item Now the triangle inequality states that for any two vectors $x, y \in \mathbb{R}^d$ it holds that 
				%
				\begin{equation*}
						\norm{x + y} \leq \norm{x} + \norm{y},
					\end{equation*} 
				%
				 so that
				%
				\begin{equation*} 
						%				\label{help_ineq_FTL_analysis_sec}
						%
						\norm{a_t - z_t} \leq \norm{a_t} + \norm{z_t} \leq 2L.
						%
						\tag{3}
					\end{equation*}
				%				
				\pause 

				%
				 \item Summing over all $t$ in (2) and using (3) we arrive at
				%
				\begin{equation*}
						%
						\begin{split}
								%
								\sum\limits_{t=1}^T \left(\l(a_t,z_t) - \l(a_{t+1},z_t)  \right) 
								%					
								 \leq \sum\limits_{t=1}^T \left(\tfrac{1}{t} \cdot \norm{a_t - z_t}^2 \right) 
								%
								&\leq  \sum\limits_{t=1}^T \tfrac{1}{t} \cdot (2L)^2 \\
								%					
								 &= 4L^2 \cdot \sum\limits_{t=1}^T \tfrac{1}{t}.
								%
								%				\leq 4L^2 \cdot \left(\log(T) + 1 \right),
								%
							\end{split}
						%
						%
					\end{equation*}
				
			\end{itemize}
	\end{itemize}
\end{frame}


\begin{frame} 
	\frametitle{FTL for OQO problems: Analysis}
	%	
	\small
	\begin{itemize}	
		%	
		\item[]
		\begin{itemize}	
				\item[]
				%			
				
				\fbox{\begin{minipage}{0.9\textwidth}	
				\begin{equation*}
						%
						\begin{split}
								%
								\mbox{\textbf{Reminder:} \quad } \sum\limits_{t=1}^T \left(\l(a_t,z_t) - \l(a_{t+1},z_t)  \right) 
								%
								\leq  4L^2 \cdot \sum\limits_{t=1}^T \tfrac{1}{t} 
								%
								%				\leq 4L^2 \cdot \left(\log(T) + 1 \right),
								%
							\end{split}
						%
						%
					\end{equation*}
				\end{minipage}
			}
				%
				\pause			
				%
				 \item Now, it holds that   $\sum\limits_{t=1}^T \tfrac{1}{t} \leq \log(T) + 1,$ so that we obtain
				%				
				 \begin{equation*}
						%
						\begin{split}
								%
								\sum\limits_{t=1}^T \left(\l(a_t,z_t) - \l(a_{t+1},z_t)  \right) 
								%					
								 &\leq 4L^2 \cdot \sum\limits_{t=1}^T \tfrac{1}{t} 
								%
								  \leq  4L^2 \cdot \left(\log(T) + 1 \right),
								%
							\end{split}
						%
						%
					\end{equation*}
				%
				which is what we wanted to prove. \qed
				%
			\end{itemize}
		%		
	\end{itemize}
	%	
\end{frame}


%
\endlecture
\end{document}
