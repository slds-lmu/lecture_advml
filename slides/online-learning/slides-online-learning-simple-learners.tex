\documentclass[11pt,compress,t,notes=noshow, xcolor=table]{beamer}
\input{../../style/preamble}
\input{../../latex-math/basic-math}
\input{../../latex-math/basic-ml}
\input{../../latex-math/ml-online}

\usepackage{multicol}

\newcommand{\titlefigure}{figure/FTL_illustration}
\newcommand{\learninggoals}{
	\item Formalization of online learning algorithms
	\item Getting to know the FTL algorithm	 
	\item See that it works for online quadratic optimzation (OQO) problems 
	%  \item Get to know the FTRL algorithm as a stable alternative
}


\title{Advanced Machine Learning}
\date{}

\begin{document}

\lecturechapter{Simple Online Learning Algorithms}
\lecture{Advanced Machine Learning}



\sloppy


\begin{frame}
	\frametitle{The online learner}
	%	
	\small
	%	
	\begin{itemize}
		%		
		\item In the following, we will consider a first (online) learner for   online learning problems. Note that a learner can be defined in a formal way.
		%
		\pause
%		
		\item Indeed, a learner within the basic online learning protocol, say \Algo, is a function $$A: \bigcup_{t=1}^T (\Zspace \times \Aspace)^t \to \Aspace$$ that returns the current action based on (the loss $\l$ and) the full history of information so far:
		%		
		$$	a_{t+1}^{\Algo} = A(z_1,a_1^{\Algo},z_2,a_2^{\Algo},\ldots,z_t,a_t^{\Algo};\l).		$$
		%
		\pause
		\item In the extended online learning scenario, where the environmental data consists of two parts, $z_t=(z_t^{(1)},z_t^{(2)}   ),$ and the {\color{red}first part is revealed before the action in $t$ is performed}, we have that
		%		
		$$	a_{t+1}^{\Algo} =  A(z_1,a_1^{\Algo},z_2,a_2^{\Algo},\ldots,z_t,a_t^{\Algo},{\color{red} z_{t+1}^{(1)}};\l)  		$$
		%		

%		
				 
		%
	\end{itemize}
	%	
\end{frame}


\begin{frame}
	\frametitle{The online learner}
	%	
	\small
	%	
	\begin{itemize}
		%	
		%		
 		\item It will be desired that the online learner admits a \emph{cheap update formula}, which is incremental, i.e., only a portion of the previous data is necessary to determine the next action.
		%
		\item  For instance, there exists a function $u:\Zspace \times \Aspace \to \Aspace$ such that 
		%
		$$	A(z_1,a_1^{\Algo},z_2,a_2^{\Algo},\ldots,z_t,a_t^{\Algo};\l) = u(z_t,a_t^{\Algo}).$$
		
		%
	\end{itemize}
	%	
\end{frame}

\begin{frame} 
	\frametitle{Follow the leader algorithm}
	%	
	\footnotesize
	%
	\begin{itemize}
		%		
		\item A simple algorithm to tackle online learning problems  is the \textbf{Follow the leader} (FTL) algorithm.
		%		
		 %\item Suppose that the online learning problem consists of an action space $\Aspace\subset \R^{d_1},$ an environmental space $\Zspace \subset \R^{d_2}$ and some loss function $\l:\Aspace \times \Zspace \to \R$ with $d_1,d_2\in \N.$
		%
		 \item The algorithm takes as its action $a_t^{\FTL} \in \Aspace$ in time step $t \geq 2,$ the element which has the minimal cumulative loss so far over the previous $t-1$ time periods:
		%
		\begin{align*} 
			%		\label{defi_ftl_estimate}
			%	
			a_t^{\FTL} \in \argmin_{a \in \mathcal{A}} \sum_{s=1}^{t-1} \l(a,z_s).
			%	
		\end{align*}
		%
		{\tiny (Technical side note: if there are more than one minimum, then one of them is chosen. Moreover, $a_1^{\FTL}$ is arbitrary. )}
		%	
		\pause
		\begin{minipage}{0.55\textwidth}	
		 \item \emph{Interpretation:} The  action $a_t^{\FTL}$ is the current ''leader'' of the actions in $\Aspace$ in time step $t$, as it has the smallest cumulative loss (error) so far.
	 \end{minipage}
		%		 
		\begin{minipage}{0.3\textwidth}	
		\begin{figure}
			\centering
			\includegraphics[width=1.2\linewidth]{figure/FTL_illustration}
%			\caption{}
%			\label{fig:ftlillustration}
		\end{figure}
	\end{minipage}
		
	\end{itemize}
	%	
\end{frame}


\begin{frame} 
	\frametitle{Follow the leader algorithm}
	%	
	\footnotesize
	
	\fbox{\begin{minipage}{0.99\textwidth}
			\begin{align*} 
		%		\label{defi_ftl_estimate}
		%	
		a_t^{\FTL} \in \argmin_{a \in \mathcal{A}} \sum_{s=1}^{t-1} \l(a,z_s).
		%	
	\end{align*}
	\end{minipage}} 
	
	%
	\begin{itemize}
		%	
		\item Note that the action selection rule of FTL is natural and has much in common with the classical batch learning approaches based on empirical risk minimization.
		
		\item 	This results in a first issue regarding the computation time for the action, because the longer we run this algorithm, the slower it becomes (in general) due to the growth of the seen data.
	\end{itemize}
	%	
\end{frame}

\begin{frame} 
	\frametitle{FTL: A Helpful Lemma}
	%	
	\small
	%
	 \textbf{Lemma:}
		%	
		Let $a_1^{\FTL}, a_2^{\FTL}, \ldots$ be the sequence of actions used by the FTL algorithm for the environmental data sequence $z_1,z_2,\ldots .$
		%	
		\pause
		
		 Then, for all $\tilde a \in \Aspace$ it holds that 
		%	
		\begin{align*}
			%	
			R_T^{\FTL}(\tilde a) 
			%	
			&= \sum\nolimits_{t=1}^T \left(\l(a_t^{\FTL},z_t) - \l(\tilde a,z_t)\right) \\
%			
			& \leq \sum\nolimits_{t=1}^T \left(\l(a_t^{\FTL},z_t) - \l(a_{t+1}^{\FTL},z_t)\right)\\
%			
			&= \sum\nolimits_{t=1}^T \l(a_t^{\FTL},z_t) -  \sum\nolimits_{t=1}^T  \l(a_{t+1}^{\FTL},z_t) .
			%	
		\end{align*}
		%	
		\pause 
		 In particular,
		%
		\begin{equation*}
			%	
			R_T^{\FTL} {\color{green} \leq}    {\color{blue} \sum\nolimits_{t=1}^T \l(a_t^{\FTL},z_t) }-  {\color{red}\sum\nolimits_{t=1}^T  \l(a_{t+1}^{\FTL},z_t)}
			%	
		\end{equation*}
		% 
		\pause
		 \emph{Interpretation}: the regret of the FTL algorithm is {\color{green}bounded} by {\color{blue} the difference of cumulated losses of itself} compared to {\color{red} its one-step lookahead cheater version}.
		%
	%	
\end{frame}

\begin{frame} 
	\frametitle{FTL: A Helpful Lemma}
	%	
	\small
	%
		%		
		\textbf{Proof:}
		%
		%			
		In the following, we denote  $a_1^{\FTL}, a_2^{\FTL}, \ldots$  simply by  $a_1, a_2, \ldots$  	
		\pause
		
		%
		 First, note that the assertion can be restated as follows
		\begin{equation*}
			%
			\begin{split}
			R_T^{\FTL}(\tilde a) = &\sum\limits_{t=1}^T \left(\l(a_t,z_t) - \l(\tilde a,z_t)\right) \leq \sum\limits_{t=1}^T \left(\l(a_t,z_t) - \l(a_{t+1},z_t)\right) \\
				%&\Leftrightarrow - \sum\limits_{t=1}^T \l(\tilde a,z_t) \leq - \sum\limits_{t=1}^T \l(a_{t+1},z_t) \\
				 &\Leftrightarrow \sum\limits_{t=1}^T \l(a_{t+1},z_t) \leq  \sum\limits_{t=1}^T \l(\tilde a,z_t).
			\end{split} 
			%	
		\end{equation*}
		%
		\pause
		 Hence, we will verify the inequality $\sum\nolimits_{t=1}^T \l(a_{t+1},z_t) \leq  \sum\nolimits_{t=1}^T \l(\tilde a,z_t),$ which implies the assertion.
		 \lz
		 
		%			 
		 $\leadsto$ This will be done  by induction over $T$.
		%			
	%	
\end{frame}

\begin{frame} 
	\frametitle{FTL: A Helpful Lemma}
	%	
	\small
			\fbox{\begin{minipage}{0.99\textwidth}
				\begin{align*} 
					%		\label{defi_ftl_estimate}
					%	
					\mbox{\textbf{Reminder:} \qquad } a_t^{\FTL} \in \argmin_{a \in \mathcal{A}} \sum_{s=1}^{t-1} \l(a,z_s).
					%	
				\end{align*}
			\end{minipage}} 
	%
			\textbf{Initial step: $T=1.$} It holds that
			%
			\begin{equation*}
				%
				\begin{split}
					%
					\sum\limits_{t=1}^T \l(a_{t+1},z_t) 
					%					
					 &= \l(a_2,z_1) 
					 = \l\left(\arg\min\limits_{a \in \Aspace} \l(a,z_1),z_1   \right) \\ 
					%				
					 &= \min\limits_{a \in \Aspace} \l(a,z_1) \leq \l(\tilde a,z_1)  \quad \left(= \sum\nolimits_{t=1}^T \l(\tilde a,z_t) \right)
					%				\quad \forall \tilde a \in \Aspace.
					%
				\end{split} 
				%
			\end{equation*} 
			%
			for all $\tilde a  \in \Aspace$. 
			
			%	
			{\visible<2>{ \textbf{Induction Step: $T -1 \rightarrow T.$}
			%
			Assume  that for any $\tilde a  \in \Aspace$ it holds that 
			%
			\begin{equation*}
				%
				\sum\nolimits_{t=1}^{T-1} \l(a_{t+1},z_t) \leq  \sum\nolimits_{t=1}^{T-1} \l(\tilde a,z_t).
				%
			\end{equation*}
			%
			 Then, the following holds as well (adding $\l(a_{T+1},z_T)$ on both sides)
			%
			\begin{equation*}
				%
				\begin{split}
					%	
					%		&\quad \sum\limits_{t=1}^{T-1} \l(a_{t+1},z_t) \leq  \sum\limits_{t=1}^{T-1} \l(\tilde a,z_t) \\
					%		&\Leftrightarrow \l(a_{t+1},z_t) + \sum\limits_{t=1}^{T-1} \l(a_{t+1},z_t) \leq  \l(a_{t+1},z_t) + \sum\limits_{t=1}^{T-1} \l(\tilde a,z_t)\\
					%		&\Leftrightarrow 
					\sum\nolimits_{t=1}^{T} \l(a_{t+1},z_t) \leq \l(a_{T+1},z_T) + \sum\nolimits_{t=1}^{T-1} \l(\tilde a,z_t), \quad \forall \tilde a  \in \Aspace.
					%		
				\end{split}
				%
			\end{equation*}}}
			
%		\end{itemize}
%	\end{itemize}
	%	
\end{frame}

\begin{frame} 
	\frametitle{FTL: A Helpful Lemma}
	%	
	\small
	%
			\fbox{\begin{minipage}{0.99\textwidth}
			\begin{align*}
				%
				\begin{split}
					%	
					&\mbox{\textbf{Reminder (1):} \qquad }\sum\nolimits_{t=1}^{T} \l(a_{t+1},z_t) \leq \l(a_{T+1},z_T) + \sum\nolimits_{t=1}^{T-1} \l(\tilde a,z_t). \\
%					
					&\mbox{\textbf{Reminder (2):} \qquad } a_t^{\FTL} \in \argmin_{a \in \mathcal{A}} \sum_{s=1}^{t-1} \l(a,z_s).
					%		
				\end{split}
				%
			\end{align*} 
		\end{minipage}} 
			
			{\visible<2>{  Using (1) with  $\tilde a = a_{T+1}$ yields
			%		
			\begin{align*}\allowdisplaybreaks
				%\begin{split}
				%
				%	&\quad \sum\limits_{t=1}^{T} \l(a_{t+1},z_t) \leq \l(a_{T+1},z_T) + \sum\limits_{t=1}^{T-1} \l(\tilde a,z_t) \\
				%%	
				%	&\Leftrightarrow \sum\limits_{t=1}^{T} \l(a_{t+1},z_t) \leq \l(a_{T+1},z_T) + \sum\limits_{t=1}^{T-1} \l(a_{T+1},z_t) \\
				%%	
				%	&\Leftrightarrow 
				\sum\limits_{t=1}^{T} \l(a_{t+1},z_t) 
				%				
				&\leq \sum\limits_{t=1}^{T} \l(a_{T+1},z_t) 
				 = \sum\limits_{t=1}^{T} \l\Big(\arg\min\limits_{a \in \mathcal{A}} \sum\limits_{t=1}^T \l(a,z_t),z_t\Big) \\
				%	
				%	&\Leftrightarrow \sum\limits_{t=1}^{T} \l(a_{t+1},z_t)
				 &= \min\limits_{a \in \mathcal{A}}  \sum\limits_{t=1}^{T}\l(a,z_t) 
				%	 
				 \leq \sum\limits_{t=1}^{T}\l(\tilde a,z_t)
				%\end{split}
			\end{align*} 
			%
			for all $\tilde a \in \Aspace.$
			%	This completes the proof.
			\qed }}
			%			
			%			
			%			
			%			
			%			
		
	%	
\end{frame}




\begin{frame} 
	\frametitle{FTL for OQO problems}
	%	
	\footnotesize
%
	\begin{itemize}
%		
		\item One popular instantiation of the online learning problem is the problem of \emph{online quadratic optimization} (OQO).
		%		
		\item In its most general form, the loss function is thereby defined as 
		%	
		\begin{align*}
			%		 \label{defi_loss_quadratic_optim}
			%		
			\l(a_t,z_t) =  \frac12 \norm{a_t-z_t}^2,
			%		
		\end{align*}
%	
		where $\Aspace,\Zspace \subset \R^d.$ 
%		
		%		
	 	\item  {\visible<2->{ \textbf{Proposition:}
		%	
		Using FTL on any online quadratic optimization problem with $\Aspace = \mathbb{R}^d$ and $V = \sup\limits_{z \in \Zspace} \norm{z}$, leads to a regret of  
		%	 
		\begin{equation*}
			%	
			R_T^{\FTL}  \leq 4V^2 \, (\log(T) + 1 ).
			%		
		\end{equation*} }}
		%
	 	\item {\visible<3>{ This result is satisfactory for three reasons:
%		
		\begin{enumerate}\footnotesize
			%			
			\item  The regret is definitely sublinear, that is,	$R_T^{\FTL}  = o(T).$
			%			
			\item  We just have a mild constraint on the online quadratic optimization problem, namely that $
			\norm{z} \leq V$ holds for any possible environmental data instance $z\in \Zspace.$
			%			
			\item The action $a_t^{\FTL}$ is simply the empirical average of the environmental data seen so far: $a_t^{\FTL} = \frac{1}{t-1}\sum_{s=1}^{t-1} z_s.$
			%		
		\end{enumerate}   }}
		%	
		
	\end{itemize}
	%	
\end{frame}


%\begin{frame} 
%	\frametitle{FTL for OQO problems: Theoretical guarantees}
%	%	
%	\small
%	\begin{itemize}		
%		
%		\item \textbf{Proof:}
%		%	
%		\begin{itemize}
%			%				
%			\item In the following, we denote  $a_1^{\FTL}, a_2^{\FTL}, \ldots$  simply by  $a_1, a_2, \ldots$  	
%			%			
%			 \item  Using the {\color{blue} lemma} for the comparison of FTL with its one-step lookahead cheater version, we just have to show that
%			%
%			\begin{align} \label{ineq_help_FTL_quadr}
%				%	
%				\sum\limits_{t=1}^T \left(\l(a_t,z_t) - \l(a_{t+1},z_t)  \right) \leq 4L^2 \cdot \left(\log(T) + 1 \right),
%				%	
%			\end{align}
%			%						
%			since $R_T^{\FTL} \stackrel{{\color{blue} \mbox{Lemma}}}{\leq} \sum\nolimits_{t=1}^T \left(\l(a_t,z_t) - \l(a_{t+1},z_t)  \right).$ 
%			%				
%			 \item So, we will prove (\ref{ineq_help_FTL_quadr}).
%			%				
%			For this purpose, we compute the explicit form of the actions of FTL for this type of online learning problem.
%			%				
%		\end{itemize}
%		%		
%	\end{itemize}
%\end{frame}
%
%
%\begin{frame} 
%	\frametitle{FTL for OQO problems: Theoretical guarantees}
%	%	
%	\small
%	\begin{itemize}	
%		%	
%		\item[]
%		\begin{itemize}	
%			%				
%			\item Claim: It holds that $a_t = \frac{1}{t - 1} \cdot \sum\nolimits_{s = 1}^{t-1} z_s,$ if $\l(a,z)=\frac12\norm{a-z}^2 .$ 
%			%				
%			\begin{itemize}
%				%				
%				 \item	Recall that $$a_t^{\FTL} = \argmin{a \in \mathcal{A}} \sum_{s=1}^{t-1} \l(a,z_s)  = \argmin{a \in \mathcal{A}} \sum_{s=1}^{t-1} \frac12 \norm{a - z_s}^2. $$
%				%						
%				 \item So, we have to find the minimizer of the function $$f(a):= \sum_{s=1}^{t-1} \frac12 \norm{a - z_s}^2  =   \sum_{s=1}^{t-1} \frac12 (a - z_s)^\top (a-z_s).$$
%				%						
%				 \item Compute  $\nabla f(a) = \sum_{s=1}^{t-1}  a  - z_s =  (t-1) a  - \sum_{s=1}^{t-1} z_s,$ which we set to zero and solve with respect to $a$ to obtain the claim.\\
%				%
%				{\tiny (The Hessian of $f$ is semi-positive definite, so that this leads indeed to a minimizer.)}
%				%	
%				%				
%			\end{itemize} 
%			%				
%		\end{itemize}
%		%		
%	\end{itemize}
%\end{frame}
%
%
%\begin{frame} 
%	\frametitle{FTL for OQO problems: Theoretical guarantees}
%	%	
%	\small
%	\begin{itemize}	
%		%	
%		\item[]
%		\begin{itemize}	 
%			%
%			\item Hence, $a_t$ is the empirical average of $z_1, \ldots, z_{t-1}$ and we can provide the following incremental update formula for its computation
%			%
%			\begin{equation*}
%				%
%				\begin{split}
%					%	
%					a_{t+1} &= \frac{1}{t} \cdot \sum\limits_{s = 1}^{t} z_s
%					%	
%					 = \frac{1}{t}(z_t + (t-1) a_t) 
%					%	
%					 =  \frac{1}{t} z_t + \left(1 - \frac{1}{t} \right)  a_t.
%					%	
%				\end{split}
%				%
%			\end{equation*} 
%			%
%			%
%			 \item From the last display we derive that
%			%
%			\begin{equation*}
%				%	
%				\begin{split}
%					% 
%					a_{t+1} - z_t = \left(1 - \frac{1}{t} \right) \cdot a_t + \frac{1}{t} z_t - z_t = \left(1 - \frac{1}{t} \right) \cdot (a_t - z_t).
%					%
%				\end{split}
%				%	
%			\end{equation*}
%			%			
%			 \item Claim:
%			%
%			\begin{align*} 
%				%				\label{help_ineq_FTL_analysis}
%				%	
%				\l(a_t,z_t) - \l(a_{t+1},z_t) \leq \frac{1}{t} \cdot \norm{a_t - z_t}^2. \tag{2}
%				%	
%			\end{align*}
%			%	
%		\end{itemize}
%		%
%	\end{itemize}
%\end{frame}
%
%
%\begin{frame} 
%	\frametitle{FTL for OQO problems: Theoretical guarantees}
%	%	
%	\small
%	\begin{itemize}	
%		%	
%		\item[]
%		\begin{itemize}	 
%			%
%			Indeed, this can be seen as follows
%			%
%			{\footnotesize
%				\begin{align*}
%					%
%					\l(a_t,z_t) - \l(a_{t+1},z_t) 
%					%
%					&= \frac{1}{2}\norm{a_t - z_t}^2 - \frac{1}{2}\norm{a_{t+1} - z_{t}}^2 \\
%					%
%					 &= \frac{1}{2} \left( \norm{a_t - z_t}^2 - \norm{a_{t+1} - z_{t}}^2 \right) \\
%					%
%					%				\Big[\mbox{Using \ }a_{t+1} - z_t  = \left(1 - \frac{1}{t} \right) \cdot (a_t - z_t) \Big] \quad  
%					 &= \frac{1}{2} \left( \norm{a_t - z_t}^2  - \norm{\left(1 - \frac{1}{t} \right) \cdot (a_t - z_t)}^2 \right) \\
%					%
%					 &= \frac{1}{2} \left( \norm{a_t - z_t}^2  - \left(1 - \frac{1}{t} \right)^2 \cdot \norm{a_t - z_t}^2 \right) \\
%					%
%					 &= \frac{1}{2} \left(1 - \left(1 - \frac{1}{t} \right)^2 \right) \cdot \norm{a_t - z_t}^2 \\
%					%
%					 &= \left(\frac{1}{t} - \frac{1}{2t^2}\right) \cdot \norm{a_t - z_t}^2 \\
%					%
%					 &\leq \frac{1}{t} \cdot \norm{a_t - z_t}^2.
%					%
%				\end{align*}	
%			}
%			%
%		\end{itemize}
%	\end{itemize}
%\end{frame}
%
%
%\begin{frame} 
%	\frametitle{FTL for OQO problems: Theoretical guarantees}
%	%	
%	\small
%	\begin{itemize}	
%		%	
%		\item[]
%		\begin{itemize}	
%			\footnotesize
%			
%			\item Since by assumption $L = \sup\limits_{z \in \Zspace} \norm{z}$ and $a_t$ is the empirical average of $z_1, \ldots, z_{t-1}$, we have that $\norm{a_t} \leq L.$
%			%
%			 \item Now the triangle inequality states that for any two vectors $x, y \in \mathbb{R}^d$ it holds that 
%			%
%			\begin{equation*}
%				\norm{x + y} \leq \norm{x} + \norm{y},
%			\end{equation*} 
%			%
%			 so that
%			%
%			\begin{equation*} 
%				%				\label{help_ineq_FTL_analysis_sec}
%				%
%				\norm{a_t - z_t} \leq \norm{a_t} + \norm{z_t} \leq 2L.
%				%
%				\tag{3}
%			\end{equation*}
%			%				 
%			 \begin{align}
%				%				 \label{help_ineq_FTL_analysis}
%				%	
%				\mbox{\textbf{Reminder:} \quad }\l(a_t,z_t) - \l(a_{t+1},z_t) \leq \frac{1}{t} \cdot \norm{a_t - z_t}^2 \tag{2}.
%				%	
%			\end{align}	 
%			%
%			 \item Summing over all $t$ in (2) and using (3) we arrive at
%			%
%			\begin{equation*}
%				%
%				\begin{split}
%					%
%					\sum\limits_{t=1}^T \left(\l(a_t,z_t) - \l(a_{t+1},z_t)  \right) 
%					%					
%					 &\leq \sum\limits_{t=1}^T \left(\frac{1}{t} \cdot \norm{a_t - z_t}^2 \right) 
%					%
%					 \leq  \sum\limits_{t=1}^T \frac{1}{t} \cdot (2L)^2 
%					%					
%					 = 4L^2 \cdot \sum\limits_{t=1}^T \frac{1}{t}.
%					%
%					%				\leq 4L^2 \cdot \left(\log(T) + 1 \right),
%					%
%				\end{split}
%				%
%				%
%			\end{equation*}
%			
%		\end{itemize}
%	\end{itemize}
%\end{frame}
%
%
%\begin{frame} 
%	\frametitle{FTL for OQO problems: Theoretical guarantees}
%	%	
%	\small
%	\begin{itemize}	
%		%	
%		\item[]
%		\begin{itemize}	
%			\item[]
%			%				
%			\begin{equation*}
%				%
%				\begin{split}
%					%
%					\mbox{\textbf{Reminder:} \quad } \sum\limits_{t=1}^T \left(\l(a_t,z_t) - \l(a_{t+1},z_t)  \right) 
%					%
%					\leq  4L^2 \cdot \sum\limits_{t=1}^T \frac{1}{t} 
%					%
%					%				\leq 4L^2 \cdot \left(\log(T) + 1 \right),
%					%
%				\end{split}
%				%
%				%
%			\end{equation*}
%			%			
%			%
%			 \item Now, it holds that   $\sum\limits_{t=1}^T \frac{1}{t} \leq \log(T) + 1,$ so that we obtain
%			%				
%			 \begin{equation*}
%				%
%				\begin{split}
%					%
%					\sum\limits_{t=1}^T \left(\l(a_t,z_t) - \l(a_{t+1},z_t)  \right) 
%					%					
%					 &\leq 4L^2 \cdot \sum\limits_{t=1}^T \frac{1}{t} 
%					%
%					  \leq  4L^2 \cdot \left(\log(T) + 1 \right),
%					%
%				\end{split}
%				%
%				%
%			\end{equation*}
%			%
%			which is what we wanted to prove. \qed
%			%
%		\end{itemize}
%		%		
%	\end{itemize}
%	%	
%\end{frame}




%
\endlecture
\end{document}
