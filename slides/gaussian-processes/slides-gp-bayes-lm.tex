\documentclass[11pt,compress,t,notes=noshow, xcolor=table]{beamer}
\input{../../style/preamble}
\input{../../latex-math/basic-math}
\input{../../latex-math/basic-ml}
\input{../../latex-math/ml-gp}

\newcommand{\titlefigure}{figure/bayes_lm/posterior_5_3.pdf} % does not fit
\newcommand{\learninggoals}{
  \item Know the Bayesian linear model
  \item The Bayesian LM returns a (posterior) distribution instead of a point estimate
  \item Know how to derive the posterior distribution for a Bayesian LM 
}

\title{Advanced Machine Learning}
\date{}

\begin{document}

\lecturechapter{The Bayesian Linear Model}
\lecture{Advanced Machine Learning}


\begin{vbframe}{Review: The Bayesian Linear Model}

    Let $\D = \left\{(\xi[1], \yi[1]), ..., (\xi[n], \yi[n])\right\}$ be a training set of i.i.d. observations from some unknown distribution.

    \begin{figure}
      \includegraphics[width=0.6\textwidth]{figure/bayes_lm/example.pdf}
    \end{figure}

    Let $\yv = (\yi[1], ..., \yi[n])^\top$ and $\Xmat \in \R^{n \times p+1}$ be the design matrix where the i-th row contains vector $\xi$.  
    
    \framebreak

    The linear regression model is defined as

    $$
    y = \fx + \epsilon = \thetab^T \xv + \epsilon 
    $$

    or on the data:

    \begin{eqnarray*}
    \yi &=& \fxi + \epsi = \thetab^T \xi + \epsi, \quad \text{for } i \in \{1, \ldots, n\}
    \end{eqnarray*}
    

    We now assume (from a Bayesian perspective) that also our parameter vector $\thetab$ is stochastic and follows a distribution.
    The observed values $\yi$ differ from the function values $\fxi$ by some additive noise, which is assumed to be i.i.d. Gaussian 
    $$
    \epsi \sim \mathcal{N}(0, \sigma^2)$$
    and independent of $\xv$ and $\thetab$.

    \framebreak

    Let us assume we have \textbf{prior beliefs} about the parameter $\thetab$ that are represented in a prior distribution $\thetab \sim \mathcal{N}(\zero, \tau^2 \id_p).$

    \lz 

    Whenever data points are observed, we update the parameters' prior distribution according to Bayes' rule 

    $$
    \underbrace{p(\thetab | \Xmat, \yv)}_{\text{posterior}} = \frac{\overbrace{p(\yv | \Xmat, \thetab)}^{\text{likelihood}}\overbrace{q(\thetab)}^{\text{prior}}}{\underbrace{p(\yv|\Xmat)}_{\text{marginal}}}. 
    $$

    \framebreak 

    The posterior distribution of the parameter $\thetab$ is again normal distributed (the Gaussian family is self-conjugate): 

    $$
    \thetab ~|~ \Xmat, \yv \sim \mathcal{N}(\sigma^{-2}\bm{A}^{-1}\Xmat^\top\yv, \bm{A}^{-1})
    $$

    with $\bm{A}:= \sigma^{-2}\Xmat^\top\Xmat + \frac{1}{\tau^2} \id_p$. \\
    
    Remarks: (1) Please see the Deep Dive part for the detailed derivation. (2) The expectation of $\thetab ~|~ \Xmat, \yv$ is exactly the solution of ridge regression.

    \lz 

    \begin{footnotesize}
    \textbf{Note:} If the posterior distribution $p(\thetab~|~\Xmat, \yv)$ are in the same probability distribution family as the prior $q(\thetab)$ w.r.t. a specific likelihood function $p(\yv~|~\Xmat, \thetab)$, they are called \textbf{conjugate distributions}. The prior is then called a \textbf{conjugate prior} for the likelihood. The Gaussian family is self-conjugate: Choosing a Gaussian prior for a Gaussian Likelihood ensures that the posterior is Gaussian. 
    \end{footnotesize}

    \framebreak 
    
    \begin{figure}
      \includegraphics[width=0.5\textwidth]{figure/bayes_lm/prior_1.pdf}~\includegraphics[width=0.5\textwidth]{figure/bayes_lm/prior_2.pdf}
    \end{figure}
    
    \framebreak 

    \foreach \x in{5, 10, 20} {
        \begin{figure}
            \includegraphics[width=0.49\textwidth]{figure/bayes_lm/posterior_\x_1.pdf}~  \includegraphics[width=0.49\textwidth]{figure/bayes_lm/posterior_\x_2.pdf}
        \end{figure}
    }
    
    
    \framebreak 
    
    Based on the posterior distribution 
    
    $$
    \thetab ~|~ \Xmat, \yv \sim \mathcal{N}(\sigma^{-2}\bm{A}^{-1}\Xmat^\top\yv, \bm{A}^{-1})
    $$
    
    we can derive the predictive distribution for a new observations $\xv_*$. The predictive distribution for the Bayesian linear model, i.e. the distribution of $\thetab^\top \xv_*$, is 
    
    $$
    y_* ~|~ \Xmat, \yv, \xv_* \sim \mathcal{N}(\sigma^{-2}\yv^\top \Xmat \Amat^{-1}\xv_*, \xv_*^\top\Amat^{-1}\xv_*)
    $$ 

    Please see the Deep Dive part for more details.
    
    \framebreak 
    
    
    \foreach \x in{5, 10, 20} {
    \begin{figure}
      \includegraphics[width=0.5\textwidth]{figure/bayes_lm/posterior_\x_3.pdf} \\
      \begin{footnotesize}
        For every test input $\xv_*$, we get a distribution over the prediction $y_*$. In particular, we get a posterior mean (orange) and a posterior variance (grey region equals $+/-$ two times standard deviation). 
      \end{footnotesize}
    \end{figure}
    }

\end{vbframe}


\begin{vbframe}{Summary: The Bayesian Linear Model}

\begin{itemize}
  \item By switching to a Bayesian perspective, we do not only have point estimates for the parameter $\thetab$, but whole \textbf{distributions}
  \item From the posterior distribution of $\thetab$, we can derive a predictive distribution for $y_* = \thetab^\top \xv_*$.  
  \item We can perform online updates: Whenever datapoints are observed, we can update the \textbf{posterior distribution} of $\thetab$
\end{itemize}

Next, we want to develop a theory for general shape functions, and not only for linear function. 

\end{vbframe}







% \framebreak
% 
% Let $\Xspace = \R$. In the simplest case, the features are not modified: $\phi(x) = x$.
% 
% \lz
% 
% In the above example we might assume that the function has a quadratic shape. We would project our one-dimensional features into a two-dimensional feature space via
% 
% $$
% \phi(x) = (x, x^2).
% $$
% 
% The resulting model is
% 
% $$
% f(x) = \theta^\top \phi(x) = \theta_1 x + \theta_2 x^2.
% $$
% 
% The following plots show the posterior distribution of $\theta_1, \theta_2$ for the observed data. The more observations we have the \enquote{surer} we are about the parameters value.
% 
% \framebreak
% 
% \vspace*{1cm}
% <<eval = FALSE, echo = F, fig.height=3>>=
% plots = list()
% titles = c("a)", "b)", "c)")
% nobs = c(1, 5, 50)

% for (j in 1:3) {
%   i = nobs[j]
%   A = 1 / sigma^2 * t(as.matrix(d[1:i, 1:2])) %*% as.matrix(d[1:i, 1:2]) + diag(c(1, 1))
%   probs$posterior = mvtnorm::dmvnorm(data.grid, mean = 1 / sigma^2 * solve(A) %*% t(d[1:i, c("x1", "x2")])
% d[1:i, ]$y, sigma = solve(A))
%   p = ggplot(probs, aes(x = theta1, y = theta2, z = posterior)) + geom_contour( colour = i)
%   p = p + coord_fixed(xlim = c(-2, 2), ylim = c(-2, 2), ratio = 1)
%   p = p + xlab(expression(theta[1])) + ylab(expression(theta[2]))
%   p = p + geom_raster(aes(fill = posterior)) + geom_contour(colour = "white", bins = 5)
%   p = p + guides(fill = FALSE) + ggtitle(titles[j])
%   p

%   plots[[paste("n = ", i, sep = "")]] = p
% }

% do.call("grid.arrange", c(plots, ncol = 3))
% @
% % \vspace*{-0.6cm}
% % 
% % \begin{center}
% % \begin{footnotesize}
% % Contour lines of the posterior distribution of $(\theta_0, \theta_1)$ after a) 1 observation, b) 5 observations and c) 50 observations.
% % \end{footnotesize}
% % \end{center}

% \end{vbframe}

\endlecture
\end{document}
