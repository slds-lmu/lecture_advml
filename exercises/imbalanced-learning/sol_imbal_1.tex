\documentclass[a4paper]{article}
\usepackage[]{graphicx}\usepackage[]{xcolor}
% maxwidth is the original width if it is less than linewidth
% otherwise use linewidth (to make sure the graphics do not exceed the margin)
\makeatletter
\def\maxwidth{ %
  \ifdim\Gin@nat@width>\linewidth
    \linewidth
  \else
    \Gin@nat@width
  \fi
}
\makeatother

\definecolor{fgcolor}{rgb}{0.345, 0.345, 0.345}
\newcommand{\hlnum}[1]{\textcolor[rgb]{0.686,0.059,0.569}{#1}}%
\newcommand{\hlstr}[1]{\textcolor[rgb]{0.192,0.494,0.8}{#1}}%
\newcommand{\hlcom}[1]{\textcolor[rgb]{0.678,0.584,0.686}{\textit{#1}}}%
\newcommand{\hlopt}[1]{\textcolor[rgb]{0,0,0}{#1}}%
\newcommand{\hlstd}[1]{\textcolor[rgb]{0.345,0.345,0.345}{#1}}%
\newcommand{\hlkwa}[1]{\textcolor[rgb]{0.161,0.373,0.58}{\textbf{#1}}}%
\newcommand{\hlkwb}[1]{\textcolor[rgb]{0.69,0.353,0.396}{#1}}%
\newcommand{\hlkwc}[1]{\textcolor[rgb]{0.333,0.667,0.333}{#1}}%
\newcommand{\hlkwd}[1]{\textcolor[rgb]{0.737,0.353,0.396}{\textbf{#1}}}%
\let\hlipl\hlkwb

\usepackage{framed}
\makeatletter
\newenvironment{kframe}{%
 \def\at@end@of@kframe{}%
 \ifinner\ifhmode%
  \def\at@end@of@kframe{\end{minipage}}%
  \begin{minipage}{\columnwidth}%
 \fi\fi%
 \def\FrameCommand##1{\hskip\@totalleftmargin \hskip-\fboxsep
 \colorbox{shadecolor}{##1}\hskip-\fboxsep
     % There is no \\@totalrightmargin, so:
     \hskip-\linewidth \hskip-\@totalleftmargin \hskip\columnwidth}%
 \MakeFramed {\advance\hsize-\width
   \@totalleftmargin\z@ \linewidth\hsize
   \@setminipage}}%
 {\par\unskip\endMakeFramed%
 \at@end@of@kframe}
\makeatother

\definecolor{shadecolor}{rgb}{.97, .97, .97}
\definecolor{messagecolor}{rgb}{0, 0, 0}
\definecolor{warningcolor}{rgb}{1, 0, 1}
\definecolor{errorcolor}{rgb}{1, 0, 0}
\newenvironment{knitrout}{}{} % an empty environment to be redefined in TeX

\usepackage{alltt}
\newcommand{\SweaveOpts}[1]{}  % do not interfere with LaTeX
\newcommand{\SweaveInput}[1]{} % because they are not real TeX commands
\newcommand{\Sexpr}[1]{}       % will only be parsed by R




\usepackage[utf8]{inputenc}
%\usepackage[ngerman]{babel}
\usepackage{a4wide,paralist}
\usepackage{amsmath, amssymb, xfrac, amsthm}
\usepackage{dsfont}
%\usepackage[usenames,dvipsnames]{xcolor}
\usepackage{xcolor}
\usepackage{amsfonts}
\usepackage{graphicx}
\usepackage{caption}
\usepackage{subcaption}
\usepackage{framed}
\usepackage{multirow}
\usepackage{bytefield}
\usepackage{csquotes}
\usepackage[breakable, theorems, skins]{tcolorbox}
\usepackage{hyperref}
\usepackage{cancel}
\usepackage{bm}
\usepackage{algorithm}
\usepackage{algpseudocode}


\input{../../style/common}

\tcbset{enhanced}

\DeclareRobustCommand{\mybox}[2][gray!20]{%
	\iffalse
	\begin{tcolorbox}[   %% Adjust the following parameters at will.
		breakable,
		left=0pt,
		right=0pt,
		top=0pt,
		bottom=0pt,
		colback=#1,
		colframe=#1,
		width=\dimexpr\linewidth\relax,
		enlarge left by=0mm,
		boxsep=5pt,
		arc=0pt,outer arc=0pt,
		]
		#2
	\end{tcolorbox}
	\fi
}

\DeclareRobustCommand{\myboxshow}[2][gray!20]{%
%	\iffalse
	\begin{tcolorbox}[   %% Adjust the following parameters at will.
		breakable,
		left=0pt,
		right=0pt,
		top=0pt,
		bottom=0pt,
		colback=#1,
		colframe=#1,
		width=\dimexpr\linewidth\relax,
		enlarge left by=0mm,
		boxsep=5pt,
		arc=0pt,outer arc=0pt,
		]
		#2
	\end{tcolorbox}
%	\fi
}


%exercise numbering
\renewcommand{\theenumi}{(\alph{enumi})}
\renewcommand{\theenumii}{\roman{enumii}}
\renewcommand\labelenumi{\theenumi}


\font \sfbold=cmssbx10

\setlength{\oddsidemargin}{0cm} \setlength{\textwidth}{16cm}


\sloppy
\parindent0em
\parskip0.5em
\topmargin-2.3 cm
\textheight25cm
\textwidth17.5cm
\oddsidemargin-0.8cm
\pagestyle{empty}

\newcommand{\kopf}[2]{
\hrule
\vspace{.15cm}
\begin{minipage}{\textwidth}
%akwardly i had to put \" here to make it compile correctly
	{\sf\bf Introduction to Machine Learning \hfill Exercise sheet #1\\
	 \url{https://slds-lmu.github.io/i2ml/} \hfill #2}
\end{minipage}
\vspace{.05cm}
\hrule
\vspace{1cm}}

\newcommand{\kopfic}[2]{
\hrule
\vspace{.15cm}
\begin{minipage}{\textwidth}
%akwardly i had to put \" here to make it compile correctly
	{\sf\bf Introduction to Machine Learning \hfill Live Session #1\\
	 \url{https://slds-lmu.github.io/i2ml/} \hfill #2}
\end{minipage}
\vspace{.05cm}
\hrule
\vspace{1cm}}

\newcommand{\kopficsl}[2]{
\hrule
\vspace{.15cm}
\begin{minipage}{\textwidth}
%akwardly i had to put \" here to make it compile correctly
	{\sf\bf Supervised Learning \hfill Live Session #1\\
	 \url{https://slds-lmu.github.io/i2ml/} \hfill #2}
\end{minipage}
\vspace{.05cm}
\hrule
\vspace{1cm}}

\newcommand{\kopfaml}[2]{
\hrule
\vspace{.15cm}
\begin{minipage}{\textwidth}
%akwardly i had to put \" here to make it compile correctly
	{\sf\bf Advanced Machine Learning \hfill Exercise sheet #1\\
	 \url{https://slds-lmu.github.io/i2ml/} \hfill #2}
\end{minipage}
\vspace{.05cm}
\hrule
\vspace{1cm}}


\newcommand{\kopfdive}[1]{
\hrule
\vspace{.15cm}
\begin{minipage}{\textwidth}
%akwardly i had to put \" here to make it compile correctly
	{\sf\bf Supervised Learning \hfill Deep Dive\\
	 \url{https://slds-lmu.github.io/i2ml/} \hfill #1}
\end{minipage}
\vspace{.05cm}
\hrule
\vspace{1cm}}

\newcommand{\kopfsl}[2]{
\hrule
\vspace{.15cm}
\begin{minipage}{\textwidth}
%akwardly i had to put \" here to make it compile correctly
	{\sf\bf Supervised Learning \hfill Exercise sheet #1\\
	 \url{https://slds-lmu.github.io/i2ml/} \hfill #2}
\end{minipage}
\vspace{.05cm}
\hrule
\vspace{1cm}}

\newenvironment{allgemein}
	{\noindent}{\vspace{1cm}}

\newcounter{aufg}
\newenvironment{aufgabe}[1]
	{\refstepcounter{aufg}\textbf{Exercise \arabic{aufg}: #1}\\ \noindent}
	{\vspace{0.5cm}}

\newcounter{loes}
\newenvironment{loesung}[1]
	{\refstepcounter{loes}\textbf{Solution \arabic{loes}: #1}\\\noindent}
	{\bigskip}

\newenvironment{bonusaufgabe}
	{\refstepcounter{aufg}\textbf{Exercise \arabic{aufg}*\footnote{This
	is a bonus exercise.}:}\\ \noindent}
	{\vspace{0.5cm}}

\newenvironment{bonusloesung}
	{\refstepcounter{loes}\textbf{Solution \arabic{loes}*:}\\\noindent}
	{\bigskip}



\begin{document}
% !Rnw weave = knitr



\input{../../latex-math/basic-math.tex}
\input{../../latex-math/basic-ml.tex}

\kopfaml{8}{Imbalanced Learning}

\loesung{Performance Measures}{

\begin{enumerate}
	%
	\item Given the following confusion matrices 
%	
	$$	M_1 = \begin{pmatrix}
		0 & 10 \\
		0 & 990
	\end{pmatrix}, \quad 	
	M_2 = \begin{pmatrix}
		10 & 0 \\
		10 & 980
	\end{pmatrix},  \quad 	
	M_3 = \begin{pmatrix}
	10 & 10 \\
	0 & 980
	\end{pmatrix},	$$
%
	each of which corresponds to a classifier.
%	
	Compute the accuracy, $F_1$ score, G measure/mean, BAC and MCC of each classifier.
	
	\textbf{Accuracy:}
	
	Accuracy is computed by $\rho_{ACC} = \frac{TP+TN}{TP+TN+FN+FP}.$ 
%	
	Thus, we get
%	
	$$		\rho_{ACC}(M_1) = \frac{990}{1000} = 0.99  , \quad \rho_{ACC} (M_2)   =  \frac{990}{1000} = 0.99  ,\quad \rho_{ACC}(M_3)  =    \frac{990}{1000} = 0.99   .	$$
	
	\textbf{$F_1$ score:}
	
	$F_1$ score is computed by $ \rho_{F_1} =  2 \cdot \frac{\rho_{PPV} \cdot \rho_{TPR} }{\rho_{PPV} + \rho_{TPR}},$
	where $\rho_{PPV} = \frac{TP}{TP + FP}$ and $\rho_{TPR} = \frac{TP}{TP + FN}.$
%	
	Thus, we get for the precision (PPV)
%	
	$$		\rho_{PPV}(M_1) = \frac{0}{10} = 0  , \quad \rho_{PPV} (M_2)   =  \frac{10}{10} = 1 ,\quad \rho_{PPV}(M_3)  =    \frac{10}{20} = 1/2   ,	$$
%	
	and the recall (TPR)
%	
	$$		\rho_{TPR}(M_1) = \frac{0}{0} = 0  , \quad \rho_{TPR} (M_2)   =  \frac{10}{20} = 1/2  ,\quad \rho_{TPR}(M_3)  =    \frac{10}{10} = 1  ,	$$
%		
	so that
%
	$$		\rho_{F_1}(M_1) = 0  , \quad \rho_{F_1} (M_2)   =  2 \cdot \frac{1 \cdot 0.5}{1 + 0.5} = 2/3  ,\quad \rho_{F_1}(M_3)  =  2 \cdot \frac{1 \cdot 0.5}{1 + 0.5} = 2/3   .	$$
	
	\emph{Note:} Here, we used the convention that $0/0 = 0.$  
	
	
	\textbf{G score:}
	
	G score is computed by $ \rho_{G} =  \sqrt{\rho_{PPV} \cdot \rho_{TPR} }.$
%	
	Thus, with the above
%	
	$$		\rho_{G}(M_1) = 0  , \quad \rho_{G} (M_2)   = \sqrt{1 \cdot 0.5} = \sqrt{0.5} \approx 0.7071  ,\quad \rho_{G}(M_3)  = \sqrt{1 \cdot 0.5} = \sqrt{0.5} \approx 0.7071.	$$
	
	
	\textbf{G mean:}
	
	G mean is computed by $ \rho_{G_m} =  \sqrt{\rho_{TNR} \cdot \rho_{TPR} },$ where $\rho_{TNR} = \frac{TN}{TN + FP}$ and $\rho_{TPR} = \frac{TP}{TP + FN}.$
%	
	For the TNR we have
	%	
	$$		\rho_{TNR}(M_1) = \frac{990}{1000} = 0.99  , \quad \rho_{TNR} (M_2)   =  \frac{980}{980} = 1  ,\quad \rho_{TNR}(M_3)  =    \frac{980}{990} \approx 0.9899  ,	$$
	%	
	Thus, with the above
	%	
	$$		\rho_{G_m}(M_1) = 0, \quad \rho_{G_m} (M_2)   = \sqrt{1 \cdot 0.5} = \sqrt{0.5} \approx 0.7071  ,\quad \rho_{G_m}(M_3)  = \sqrt{1 \cdot \frac{980}{990} } \approx 0.9949.	$$
	
	
	\textbf{BAC:}
	
	BAC is computed by $ \rho_{BAC} =  \frac{\rho_{TNR} + \rho_{TPR} }{2}.$	
%	
	Thus, with the above
	%	
	$$		\rho_{BAC}(M_1) = \frac{0 + 990/1000}{2} \approx 0.5, \quad \rho_{BAC} (M_2)   = \frac{1 + 0.5}{2} = 0.75,\quad \rho_{BAC}(M_3)  = \frac{1 + \frac{980}{990} }{2} \approx 0.9949.	$$
%	
	
	\textbf{MCC:}
%	
	MCC is computed by $ \rho_{MCC} =  \frac{TP\cdot TN - FP\cdot FN}{ \sqrt{(TP+FN)(TP+FP)(TN+FN)(TN+FP)} }.$	
	%	
	Thus, with the above
	%	
	$$		\rho_{MCC}(M_1) = 0, \quad \rho_{MCC} (M_2)   = \frac{ 10\cdot 980 - 0 }{  \sqrt{20\cdot10\cdot990\cdot980}  } = 0.7035265,\quad \rho_{MCC}(M_3)  = \frac{ 10\cdot 980 - 0 }{  \sqrt{20\cdot10\cdot990\cdot980}  } = 0.7035265.	$$
	
	%		
	\item What are the population counterparts of the class-specific variants in the multiclass setting of true positive rate, positive predictive value and true negative rate?
	
	\textbf{True positive rate:}
	
	In the multiclass classification setting ($\Yspace=\{1,\ldots,g\}$), the class-specific variant of the true positive rate (or rather recall) is 
%	
	$$ \rho_{TPR_i} = \frac{n(i,i)}{n_i} $$
%	
	which corresponds to the fraction of correctly classified instances $i$	among all $i$ instances.
%	
	Thus, the population counterpart is $\P(  \yh = i ~|~ y=i ),$ where $\yh$ is the prediction random variable (i.e., $\yh=f(\xv)$ for a classifier $f$) and $y$ is the random variable representing the labels.
	
	
	\textbf{Positive predictive value:}
	
	In the multiclass classification setting ($\Yspace=\{1,\ldots,g\}$), the class-specific variant of the positive predictive value (or rather precision) is 
	%	
	$$ \rho_{PPV_i} = \frac{n(i,i)}{\sum_{j=1}^g  n(i,j) }$$
	%	
	which corresponds to the fraction of correctly classified instances $i$	among all $i$ classifications.
	%	
	Thus, the population counterpart is $\P(  y= i ~|~  \yh = i).$ 
	
	
	\textbf{True negative rate:}
	
	In the multiclass classification setting ($\Yspace=\{1,\ldots,g\}$), the class-specific variant of the true negative rate is 
	%	
	$$ \rho_{TNR_i} = \frac{\sum_{j\neq i}  n(j,j) }{ n - n_i }$$
	%	
	which corresponds to the fraction of correctly classified non-$i$ instances among all non-$i$ instances.
	%	
	Thus, the population counterpart is $\P( \yh \neq i  ~|~  y \neq i ).$ 
  
\end{enumerate}
}


\loesung{Minimum Expected Cost Principle}{

    The Minimum Expected Cost Principle implies
    \begin{align*}
        p(1|\xv) C(1, 1) + p(-1 | \xv) C(1, -1) \leq p(1|\xv) C(-1, 1) + p(-1 | \xv) C(-1. -1).
    \end{align*}
    If we define $p = p(1 |\xv) = 1 - p (-1 | \xv)$. Then we have
    \begin{align*}
        p \cdot C(1, 1) + (1 - p) C(1, -1) & \leq p \cdot C(-1, 1) + (1 - p) C(-1, -1) \\
        \Rightarrow \ p[C(1, 1) + C(-1, -1) - C(1, -1) - C(-1, 1)] &\leq C(-1, -1) - C(1, -1).       
    \end{align*}
    Note that in practice, $C(1, 1)$ or $C(-1, -1)$ is substantially smaller than $C(-1, 1)$ or $C(1, -1)$. Therefore, the term in the bracket on the left side can be assumed to be negative. Hence,
    \begin{align*}
        p = p(1 | \xv) &\geq \frac{C(-1, -1) - C(1, -1)}{C(1, 1) + C(-1, -1) - C(1, -1) - C(-1, 1)} \\
            &= \frac{C(1, -1) - C(-1, -1)}{C(1, -1) + C(-1, 1) - C(1, 1) - C(-1, -1)} \\
            &= c^* .
    \end{align*}
    Using the given cost matrix, this is simplified to 
    \begin{align*}
        c^* = \frac{1}{1 + \frac{n_{-}}{n_{+}}}.
    \end{align*}
    If we recompute the optimal threshold $c^*$ on the augmented dataset with extra samples of class $-1$ added. The new $c^*$, denoted by $c_{\text{new}}^*$, will be lower. As a consequence, the classifier is more inclined to predict class $1$. This indicates that $TP$ and $FP$ may increase or be the same. Therefore, $\rho_{PPV} = \frac{TP}{TP + FP}$ can be greater or smaller, or be the same. However, $n_{+}$ is unchanged in the augmented dataset, so $\rho_{TPR} = \frac{TP}{TP + FN} = \frac{TP}{n_{+}}$ will be greater or be the same.
    
}


\loesung{MetaCost}{


Implement the MetaCost algorithm and use it with some classifier of your choice on an imbalanced data set of your choice, where the cost-matrix is given by the cost-sensitive heuristic we saw in the lecture. Compare the confusion matrices of the underlying classifier and the MetaCost classifier as well as their total costs.


\begin{knitrout}
\definecolor{shadecolor}{rgb}{0.969, 0.969, 0.969}\color{fgcolor}\begin{kframe}
\begin{alltt}
\hlcom{## --------------------------------------------}


\hlkwd{library}\hlstd{(mlr3)}

\hlkwd{set.seed}\hlstd{(}\hlnum{123}\hlstd{)}


\hlcom{# we use the spam data in the mlr3 package (spam=positive)}
\hlstd{task}        \hlkwb{=} \hlkwd{tsk}\hlstd{(}\hlstr{"spam"}\hlstd{)}
\hlstd{data}        \hlkwb{=} \hlstd{task}\hlopt{$}\hlkwd{data}\hlstd{()}
\hlstd{n_vec}       \hlkwb{=} \hlkwd{table}\hlstd{(data[,}\hlnum{1}\hlstd{])}
\hlstd{g}           \hlkwb{=} \hlkwd{length}\hlstd{(n_vec)} \hlcom{# number of classes (two of course)}

\hlcom{# we make a naive training-test split}
\hlstd{n}           \hlkwb{=} \hlkwd{round}\hlstd{(}\hlnum{0.8} \hlopt{*} \hlstd{task}\hlopt{$}\hlstd{nrow)}
\hlstd{train_set}   \hlkwb{=} \hlnum{1}\hlopt{:}\hlstd{n}
\hlstd{train_set}   \hlkwb{=} \hlkwd{sort}\hlstd{(}\hlkwd{sample}\hlstd{(task}\hlopt{$}\hlstd{nrow, n))}
\hlcom{#unique(data[train_set,1])}

\hlstd{test_set}    \hlkwb{=} \hlkwd{setdiff}\hlstd{(}\hlkwd{seq_len}\hlstd{(task}\hlopt{$}\hlstd{nrow), train_set)}

\hlcom{# we use a CART as a classifier}
\hlstd{classifier}  \hlkwb{=} \hlstr{"classif.rpart"}

\hlcom{# Create the cost matrix via the heuristic}

\hlstd{cost_mat} \hlkwb{=} \hlkwd{matrix}\hlstd{(}\hlkwd{rep}\hlstd{(}\hlnum{0}\hlstd{,g}\hlopt{*}\hlstd{g),}\hlkwc{ncol}\hlstd{=}\hlnum{2}\hlstd{)}

\hlkwa{for} \hlstd{(i} \hlkwa{in} \hlnum{1}\hlopt{:}\hlstd{g)\{}
  \hlkwa{for} \hlstd{(j} \hlkwa{in} \hlnum{1}\hlopt{:}\hlstd{g)\{}
    \hlstd{cost_mat[i,j]} \hlkwb{=} \hlkwd{max}\hlstd{(n_vec[i]}\hlopt{/}\hlstd{n_vec[j],}\hlnum{1}\hlstd{)}
  \hlstd{\}}
  \hlstd{cost_mat[i,i]}   \hlkwb{=} \hlnum{0}
\hlstd{\}}


\hlcom{#  we use CART as the black-box classifier}
\hlstd{classifier}  \hlkwb{=} \hlstr{"classif.rpart"}
\hlcom{#  bag size}
\hlstd{B}           \hlkwb{=} \hlkwd{round}\hlstd{(n}\hlopt{/}\hlnum{4}\hlstd{)}
\hlcom{#  number of bagging iterations}
\hlstd{L}           \hlkwb{=} \hlnum{10}

\hlcom{# 1st phase: Bagging}

\hlstd{indices}     \hlkwb{=} \hlkwd{matrix}\hlstd{(}\hlkwd{rep}\hlstd{(}\hlnum{0}\hlstd{,B}\hlopt{*}\hlstd{L),}\hlkwc{ncol}\hlstd{=L)}

\hlstd{classifiers} \hlkwb{=} \hlkwd{c}\hlstd{()}

\hlkwa{for} \hlstd{(l} \hlkwa{in} \hlnum{1}\hlopt{:}\hlstd{L)\{}

  \hlcom{# bootstrapped data set (or rather the indices)}
  \hlstd{indices[,l]}         \hlkwb{=} \hlkwd{sort}\hlstd{(}\hlkwd{sample}\hlstd{(train_set,} \hlkwc{size}\hlstd{=B,} \hlkwc{replace} \hlstd{=} \hlnum{FALSE}\hlstd{))}
  \hlcom{# fit classifier on bootstrapped data set}
  \hlstd{learner}             \hlkwb{=} \hlkwd{lrn}\hlstd{(classifier,} \hlkwc{predict_type} \hlstd{=} \hlstr{"prob"}\hlstd{)}
  \hlstd{classifiers}         \hlkwb{=} \hlkwd{c}\hlstd{(classifiers,learner}\hlopt{$}\hlkwd{train}\hlstd{(task,} \hlkwc{row_ids} \hlstd{= indices[,l]))}

\hlstd{\}}

\hlcom{#    2nd phase: Relabeling}

\hlstd{y_tilde}           \hlkwb{=} \hlstd{task}\hlopt{$}\hlkwd{data}\hlstd{()[,}\hlnum{1}\hlstd{]}

\hlkwa{for} \hlstd{(i} \hlkwa{in} \hlstd{train_set)\{}


  \hlcom{# compute the proxy probability vector}
  \hlstd{p}               \hlkwb{=} \hlkwd{rep}\hlstd{(}\hlnum{0}\hlstd{,g)}

  \hlstd{tilde_L}         \hlkwb{=} \hlkwd{c}\hlstd{()}

  \hlcom{# get all bootstrapped data sets, where i is not included}

  \hlstd{temp}            \hlkwb{=} \hlkwd{which}\hlstd{(}\hlkwd{colSums}\hlstd{(i}\hlopt{==}\hlstd{indices)}\hlopt{==}\hlnum{0}\hlstd{)}

  \hlkwa{if} \hlstd{(}\hlkwd{length}\hlstd{(temp)}\hlopt{==}\hlnum{0}\hlstd{)\{} \hlcom{# the data point appears in each bootstrapped sample -> use all classifiers}
    \hlstd{tilde_L}       \hlkwb{=} \hlnum{1}\hlopt{:}\hlstd{L}
  \hlstd{\}} \hlkwa{else} \hlstd{\{} \hlcom{# only use classifiers which haven't seen the data point during training}
    \hlstd{tilde_L}       \hlkwb{=} \hlkwd{sort}\hlstd{(temp)}
  \hlstd{\}}


  \hlkwa{for} \hlstd{(l} \hlkwa{in} \hlstd{tilde_L)\{}
    \hlstd{prediction}    \hlkwb{=} \hlstd{classifiers[[l]]}\hlopt{$}\hlkwd{predict}\hlstd{(task,} \hlkwc{row_ids} \hlstd{= i  )}
    \hlstd{p}             \hlkwb{=} \hlstd{p} \hlopt{+} \hlstd{prediction}\hlopt{$}\hlstd{prob}
  \hlstd{\}}
  \hlcom{# average it}
  \hlstd{p}               \hlkwb{=} \hlstd{p}\hlopt{/}\hlkwd{length}\hlstd{(tilde_L)}

  \hlcom{# relabeling}
  \hlcom{# computes the estimated expected costs for predicting class 1,..,g}
  \hlstd{est_cost}        \hlkwb{=} \hlstd{cost_mat} \hlopt \hlkwd{t}\hlstd{(p)}
  \hlcom{# assign y the class with lowest estimated expected costs}
  \hlstd{y_tilde[i]}      \hlkwb{=} \hlkwd{which}\hlstd{(est_cost}\hlopt{==}\hlkwd{min}\hlstd{(est_cost))[}\hlnum{1}\hlstd{]}

\hlstd{\}}

\hlcom{# create the relabeled data set}
\hlstd{data_rel}          \hlkwb{=}  \hlstd{task}\hlopt{$}\hlkwd{data}\hlstd{()}
\hlstd{data_rel[,}\hlnum{1}\hlstd{]}      \hlkwb{=}  \hlstd{y_tilde}
\hlstd{task_new}          \hlkwb{=}  \hlstd{TaskClassif}\hlopt{$}\hlkwd{new}\hlstd{(}\hlstr{"rel_data"}\hlstd{,} \hlkwc{backend} \hlstd{= data_rel,} \hlkwc{target} \hlstd{= task}\hlopt{$}\hlstd{target_names )}

\hlkwd{print}\hlstd{(}\hlstr{"Number of relabeled data points"}\hlstd{)}
\end{alltt}
\begin{verbatim}
## [1] "Number of relabeled data points"
\end{verbatim}
\begin{alltt}
\hlkwd{print}\hlstd{(}\hlkwd{sum}\hlstd{(task_new}\hlopt{$}\hlkwd{data}\hlstd{()[,}\hlnum{1}\hlstd{]}\hlopt{!=}\hlstd{task}\hlopt{$}\hlkwd{data}\hlstd{()[,}\hlnum{1}\hlstd{]))}
\end{alltt}
\begin{verbatim}
## [1] 336
\end{verbatim}
\begin{alltt}
\hlcom{# initialize the cost-sensitive classifier}
\hlstd{meta_classifier} \hlkwb{=} \hlkwd{lrn}\hlstd{(classifier,} \hlkwc{predict_type} \hlstd{=} \hlstr{"prob"}\hlstd{)}

\hlcom{# 3rd phase: make classifier cost-sensitive}
\hlstd{meta_classifier}\hlopt{$}\hlkwd{train}\hlstd{(task_new,} \hlkwc{row_ids} \hlstd{= train_set)}

\hlcom{# predict with meta}

\hlstd{meta_pred}   \hlkwb{=} \hlstd{meta_classifier}\hlopt{$}\hlkwd{predict}\hlstd{(task_new,} \hlkwc{row_ids} \hlstd{= test_set)}

\hlcom{# train also the cost-insensitive CART}
\hlstd{learner}     \hlkwb{=} \hlkwd{lrn}\hlstd{(}\hlstr{"classif.rpart"}\hlstd{,} \hlkwc{predict_type} \hlstd{=} \hlstr{"prob"}\hlstd{)}
\hlstd{learner}\hlopt{$}\hlkwd{train}\hlstd{(task,} \hlkwc{row_ids} \hlstd{= train_set)}
\hlstd{CART_pred}   \hlkwb{=} \hlstd{learner}\hlopt{$}\hlkwd{predict}\hlstd{(task,} \hlkwc{row_ids} \hlstd{= test_set)}


\hlcom{# Confusion Matrices}

\hlstd{meta_pred}\hlopt{$}\hlstd{confusion}
\end{alltt}
\begin{verbatim}
##          truth
## response  spam nonspam
##   spam     316      33
##   nonspam   58     513
\end{verbatim}
\begin{alltt}
\hlstd{CART_pred}\hlopt{$}\hlstd{confusion}
\end{alltt}
\begin{verbatim}
##          truth
## response  spam nonspam
##   spam     308      40
##   nonspam   66     506
\end{verbatim}
\begin{alltt}
\hlcom{# Costs generated}

\hlkwd{sum}\hlstd{(meta_pred}\hlopt{$}\hlstd{confusion}\hlopt{*}\hlstd{cost_mat)}
\end{alltt}
\begin{verbatim}
## [1] 122.1914
\end{verbatim}
\begin{alltt}
\hlkwd{sum}\hlstd{(CART_pred}\hlopt{$}\hlstd{confusion}\hlopt{*}\hlstd{cost_mat)}
\end{alltt}
\begin{verbatim}
## [1] 141.4937
\end{verbatim}
\end{kframe}
\end{knitrout}
}



\end{document}
