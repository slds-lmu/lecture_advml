\documentclass[a4paper]{article}
% maxwidth is the original width if it is less than linewidth
% otherwise use linewidth (to make sure the graphics do not exceed the margin)
\makeatletter
\def\maxwidth{ %
  \ifdim\Gin@nat@width>\linewidth
    \linewidth
  \else
    \Gin@nat@width
  \fi
}
\makeatother

\usepackage[utf8]{inputenc}
\usepackage{a4wide,paralist}
\usepackage{amsmath, amssymb, xfrac, amsthm}
\usepackage{dsfont}
\usepackage{xcolor}
\usepackage{amsfonts}
\usepackage{graphicx}
\usepackage{hyperref}
\usepackage{bm}

% for showing python code blocks
\usepackage{listings}
\lstdefinestyle{pythonstyle}{
    language=Python,
    backgroundcolor=\color[rgb]{0.98,0.98,0.98},
    commentstyle=\color[rgb]{0.13,0.54,0.13},
    keywordstyle=\color[rgb]{0.2,0.29,0.37},
    numberstyle=\tiny\color{gray},
    stringstyle=\color[rgb]{0.73,0.13,0.13},
    basicstyle=\ttfamily\small,
    breakatwhitespace=false,
    breaklines=true,
    captionpos=b,
    keepspaces=true,
    showspaces=false,
    showstringspaces=false,
    showtabs=false,
    tabsize=4
}
\lstset{style=pythonstyle}


% %exercise numbering
\renewcommand{\theenumi}{(\alph{enumi})}
\renewcommand{\theenumii}{\roman{enumii}}
\renewcommand\labelenumi{\theenumi}


\font \sfbold=cmssbx10

\setlength{\oddsidemargin}{0cm} \setlength{\textwidth}{16cm}


\sloppy
\parindent0em
\parskip0.5em
\topmargin-2.3 cm
\textheight25cm
\textwidth17.5cm
\oddsidemargin-0.8cm
\pagestyle{empty}


\newcommand{\kopfaml}[2]{
\hrule
\vspace{.15cm}
\begin{minipage}{\textwidth}
%akwardly i had to put \" here to make it compile correctly
	{\sf\bf Advanced Machine Learning \hfill Exercise sheet #1\\
	 \url{https://slds-lmu.github.io/i2ml/} \hfill #2}
\end{minipage}
\vspace{.05cm}
\hrule
\vspace{1cm}}


\newenvironment{allgemein}
	{\noindent}{\vspace{1cm}}

\newcounter{aufg}
\newenvironment{aufgabe}[1]
	{\refstepcounter{aufg}\textbf{Exercise \arabic{aufg}: #1}\\ \noindent}
	{\vspace{0.5cm}}

\newcounter{loes}
\newenvironment{loesung}[1]
	{\refstepcounter{loes}\textbf{Solution \arabic{loes}: #1}\\\noindent}
	{\bigskip}

\newenvironment{bonusaufgabe}
	{\refstepcounter{aufg}\textbf{Exercise \arabic{aufg}*\footnote{This
	is a bonus exercise.}:}\\ \noindent}
	{\vspace{0.5cm}}

\newenvironment{bonusloesung}
	{\refstepcounter{loes}\textbf{Solution \arabic{loes}*:}\\\noindent}
	{\bigskip}

\usepackage{caption}
\usepackage{subcaption}


\begin{document}

\input{../../latex-math/basic-math.tex}
\input{../../latex-math/basic-ml.tex}

\kopfaml{10}{Imbalanced Learning}

\loesung{Cost Curves}{

\begin{enumerate}
    \item We can simply retrieve the FPRs and TPRs from the table (or matrix) and plot the curves.
\begin{lstlisting}
import numpy as np
import matplotlib.pyplot as plt

# The first column is FPR, the second column is TPR
FPR_TPR_1 = np.array(
    [[0.0, 0.00],
     [0.1, 0.60],
     [0.2, 0.75],
     [0.3, 0.825],
     [0.4, 0.85],
     [0.5, 0.875],
     [0.6, 0.90],
     [0.7, 0.925],
     [0.8, 0.950],
     [0.9, 0.975],
     [1.0, 1.0]])

FPR_TPR_2 = np.array(
    [[0.0, 0.00],
     [0.1, 0.2],
     [0.2, 0.4],
     [0.3, 0.6],
     [0.4, 0.8],
     [0.5, 0.925],
     [0.6, 0.96],
     [0.7, 0.98],
     [0.8, 0.99],
     [0.9, 0.995],
     [1.0, 1.00]]
)


def draw_roc_curves(fpr_tpr_1: np.ndarray, fpr_tpr_2: np.ndarray) -> None:
    fig, ax = plt.subplots(1, 1, figsize=(8, 7))
    ax.plot(fpr_tpr_1[:, 0], fpr_tpr_1[:, 1], marker='o', label='Classifier 1')
    ax.plot(fpr_tpr_2[:, 0], fpr_tpr_2[:, 1], marker='o', label='Classifier 2')

    ax.set_xlabel('False Positive Rate')
    ax.set_ylabel('True Positive Rate')
    ax.grid('on')

    plt.legend()
    plt.savefig('roc_curves.pdf', bbox_inches='tight')
\end{lstlisting}
The ROC curves are shown in Figure~\ref{fig:roc_curves}
\begin{figure}[h]
    \centering
    \includegraphics[width=0.5\textwidth]{figures/roc_curves}
    \caption{ROC Curves.}
    \label{fig:roc_curves}
\end{figure}

\item The cost is computed as 
\begin{align*}
    \rho_{MCE} = (FNR - FPR) \cdot \pi_{+} + FPR
\end{align*}
For each $(FPR, TPR)$ point in the ROC space, we draw a line of $(\pi_{+}, \rho_{MCE})$.

\begin{lstlisting}

def draw_cost_curves(fpr_tpr_1: np.ndarray, fpr_tpr_2: np.ndarray) -> None:
    fnr_1 = 1 - fpr_tpr_1[:, 1]
    fnr_2 = 1 - fpr_tpr_2[:, 1]
    # Probability of positive
    # Assume P = 50 points in [0.0, 1.0]
    pos_probs = np.linspace(0.0, 1.0, 50)

    # Compute the coefficient (FNR - FPR)
    # Shape after expansion: (N, 1), where N is the number of points in ROC space.
    fnr_minus_fpr_1 = np.expand_dims(fnr_1 - fpr_tpr_1[:, 0], 1)
    fnr_minus_fpr_2 = np.expand_dims(fnr_2 - fpr_tpr_2[:, 0], 1)

    # Costs shape: (N, P)
    costs_1 = fnr_minus_fpr_1 * np.expand_dims(pos_probs, 0) + np.expand_dims(fpr_tpr_1[:, 0], 1)
    costs_2 = fnr_minus_fpr_2 * np.expand_dims(pos_probs, 0) + np.expand_dims(fpr_tpr_2[:, 0], 1)
    # Point-wise minimum across cost lines.
    cost_curve_1 = costs_1.min(0)
    cost_curve_2 = costs_2.min(0)

    # Find the cross point of two cost curves. The following operation assumes that
    # cust_curve_1 is lower than cost_curve_2 in the beginning of the curves.
    cross_point_ind = np.flatnonzero(cost_curve_1 > cost_curve_2)[0]

    fig, ax = plt.subplots(1, 1, figsize=(8, 7))
    ax.plot(pos_probs, cost_curve_1, label='Classifier 1')
    ax.plot(pos_probs, cost_curve_2, label='Classifier_2')
    ax.vlines(
        pos_probs[cross_point_ind],
        ymin=0.0,
        ymax=cost_curve_1[cross_point_ind],
        linestyles='dashed',
        colors=['k'],
    )
    ax.set_xlabel('Probability of Positive')
    ax.set_ylabel('Error Rate')

    plt.legend()
    plt.savefig('cost_curves.pdf', bbox_inches='tight')


if __name__ == '__main__':
    draw_roc_curves(FPR_TPR_1, FPR_TPR_2)
    draw_cost_curves(FPR_TPR_1, FPR_TPR_2)

\end{lstlisting}
Then, we obtain Figure~\ref{fig:cost_curves}.

\begin{figure}[h]
    \centering
    \includegraphics[width=0.5\textwidth]{figures/cost_curves}
    \caption{Cost Curves.}
    \label{fig:cost_curves}
\end{figure}

\end{enumerate}

}

\loesung{Tomek Links}{

\begin{enumerate}
\item 
\begin{lstlisting}

import matplotlib.pyplot as plt
import numpy as np

from sklearn.datasets import make_classification
from sklearn.metrics import pairwise_distances


def find_tomek_links(x: np.ndarray, y: np.ndarray) -> np.ndarray:
    """Find Tomek Links in samples.

    Args:
        x: Data samples with shape (num_samples, num_features).
        y: Binary class labels with shape (num_samples,). 
           1 means positive class, 0 means negative class.

    Returns:
        An array with shape (num_samples, ) with binary values, 
        for which 1 means that the corresponding sample 
        belongs to a Tomek link, while 0 means that the 
        sample does not belong to any Tomek link.
    """

    num_samples = x.shape[0]
    dist = pairwise_distances(x)

    # Compute the min. pairwise distance between samples. Note that the distance
    # to each sample itself is 0 and trivial, so this case should be excluded when
    # computing the pairwise distance. To this end, we employ masked arrays,
    # which can be constructed by applying binary masks to the original array.
    # In the masks, 1 means excluding the corresponding samples when performing
    # operations e.g. min, argmin, etc.

    mask = np.eye(num_samples, dtype=bool)
    masked_dist = np.ma.array(dist, mask=mask)
    nearest_j_to_i = masked_dist.argmin(axis=1)
    # Check if the nearest neighbor of j is also i. 
    # If yes, they are mutually nearest.
    nearest_i_to_j = np.zeros_like(nearest_j_to_i)
    nearest_i_to_j[nearest_j_to_i] = np.arange(num_samples)

    is_mutually_closest = nearest_i_to_j == nearest_j_to_i
    # Tomek links also requires the pair to have different labels
    is_diff_y = y != y[nearest_j_to_i]
    is_tomek_sample = np.logical_and(is_mutually_closest, is_diff_y)

    return is_tomek_sample

\end{lstlisting}
In this function, \texttt{nearest\_i\_to\_j} and \texttt{nearest\_j\_to\_i} are used to determine if $i$-th sample and $j$-th sample are mutually closest. Figure~\ref{fig:mutually_closest_diagramm} shows an example of the process.

\begin{figure}[h]
    \centering
    \includegraphics[width=0.5\textwidth]{figures/mutually_closest_diagramm}
    \caption{Determining the Mutually Closest Pairs. The first two elements are consistent in both arrays. This is interpreted as that the 0-th sample and the 1-th sample are mutually closest.}
    \label{fig:mutually_closest_diagramm}
\end{figure}

\item 

\begin{lstlisting}
def find_kept_samples(
        x: np.ndarray,
        y: np.ndarray,
        is_tomek_sample: np.ndarray
) -> np.ndarray:
    """Given the binary array indicating which sample belongs to a Tomek link. Find out which sample should be kept. The sample should be removed if it belongs to a Tomek link and it is majority class.

    Args:
        x: Data samples with shape (num_samples, num_features).
        y: Binary class labels with values 0 or 1.
        is_tomek_sample: A binary array with shape (num_samples,), 
            in which 1 means that the sample belongs to a Tomek link.

    Returns:
        A binary array with shape (num_samples,), in which 1 means that the 
        corresponding sample should be kept, otherwise it should be removed.
    """
    num_samples = x.shape[0]
    to_be_kept = np.ones((num_samples, ), dtype=bool)

    # Define which class is majority
    num_pos_samples = y.sum()
    num_neg_samples = num_samples - num_pos_samples
    if num_pos_samples >= num_neg_samples:
        y_major = 1
    else:
        y_major = 0

    # Only remove the sample in a Tomek link if it is majority class
    is_major_tomek_sample = np.logical_and(is_tomek_sample, y == y_major)
    to_be_kept[is_major_tomek_sample] = False

    return to_be_kept
\end{lstlisting}

\item 
\begin{lstlisting}
def run_experiment() -> None:
    # Generate data
    random_state = np.random.RandomState(11)
    x, y = make_classification(
        100,
        n_features=2,
        n_redundant=0,
        n_clusters_per_class=1,
        flip_y=0.02,
        class_sep=1.0,
        weights=[0.25, 0.75],
        random_state=random_state)

    # Plot the original dataset
    fig, ax = plt.subplots(1, 1)
    x_pos = x[y == 1]
    x_neg = x[y == 0]
    ax.scatter(x_pos[:, 0], x_pos[:, 1], facecolors='none', edgecolors='r', label='Pos. samples')
    ax.scatter(x_neg[:, 0], x_neg[:, 1], facecolors='none', edgecolors='b', label='Neg. samples')
    ax.set_title('Original Dataset')
    ax.set_xlim(-3, 3)
    ax.set_ylim(-3, 3)
    plt.legend()
    plt.savefig('original_dataset.pdf', bbox_inches='tight')
    plt.clf()

    # Find out which samples belong to Tomek links.
    is_tomek_link = find_tomek_links(x, y)

    fig, ax = plt.subplots(1, 1)
    x_tomek = x[is_tomek_link]
    ax.scatter(x_pos[:, 0], x_pos[:, 1], facecolors='none', edgecolors='r', label='Pos. samples')
    ax.scatter(x_neg[:, 0], x_neg[:, 1], facecolors='none', edgecolors='b', label='Neg. samples')
    ax.scatter(
        x_tomek[:, 0],
        x_tomek[:, 1],
        facecolors='none',
        edgecolors='g',
        label='Set of Tomek-link samples')
    ax.set_title('Set of Tomek-link samples')
    ax.set_xlim(-3, 3)
    ax.set_ylim(-3, 3)
    plt.legend()
    plt.savefig('tomek_links.pdf', bbox_inches='tight')
    plt.clf()

    # Identify which samples should be kept and which should be removed
    to_be_kept = find_kept_samples(x, y, is_tomek_link)

    fig, ax = plt.subplots(1, 1)
    x_removed = x[~to_be_kept]
    x, y = x[to_be_kept], y[to_be_kept]
    x_pos, x_neg = x[y == 1], x[y == 0]
    ax.scatter(x_pos[:, 0], x_pos[:, 1], facecolors='none', edgecolors='r', label='Pos. samples')
    ax.scatter(x_neg[:, 0], x_neg[:, 1], facecolors='none', edgecolors='b', label='Neg. samples')
    ax.scatter(
        x_removed[:, 0],
        x_removed[:, 1],
        facecolors='none',
        edgecolor='k',
        linestyle='dotted',
        label='Removed samples')
    ax.set_title('Majority Samples in Tomek links Removed')
    ax.set_xlim(-3, 3)
    ax.set_ylim(-3, 3)
    plt.legend()
    plt.savefig('subsampled_dataset.pdf', bbox_inches='tight')
    plt.clf()


if __name__ == '__main__':
    run_experiment()
\end{lstlisting}

After running the script, we obain the following figures:
\begin{figure}[h]
    \centering
    \begin{subfigure}{0.32\textwidth}
        \centering
        \includegraphics[width=\textwidth]{figures/original_dataset}
        \caption{Original Dataset.}
    \end{subfigure}
    \hfill
    \begin{subfigure}{0.32\textwidth}
        \centering
        \includegraphics[width=\textwidth]{figures/tomek_links}
        \caption{Tomek Links.}
    \end{subfigure}
    \hfill
    \begin{subfigure}{0.32\textwidth}
        \centering
        \includegraphics[width=\textwidth]{figures/removed_dataset}
        \caption{Subsampled Dataset.}
    \end{subfigure}
\end{figure}

\end{enumerate}


}






\end{document}
