
%
The posterior distribution is obtained by Bayes' rule 
%
$$
\underbrace{p(\thetav | \Xmat, \yv)}_{\text{posterior}} = \frac{\overbrace{p(\yv | \Xmat, \thetav)}^{\text{likelihood}}\overbrace{q(\thetav)}^{\text{prior}}}{\underbrace{p(\yv|\Xmat)}_{\text{marginal}}}. 
$$
% 
In the Bayesian linear model we have a Gaussian likelihood: $\yv ~|~ \Xmat, \thetav \sim \mathcal{N}(\Xmat \thetav, \sigma^2 \id_n),$ i.e.,
%
\begin{align*}
%	
	p(\yv | \Xmat, \thetav) 
%	
	&\propto	\exp\biggl[-\frac{1}{2\sigma^2}(\yv - \Xmat\thetav)^\top(\yv - \Xmat\thetav)\biggr] \\
%	
	&= \exp\biggl[-\frac{\| \yv - \Xmat\thetav \|_2^2}{2\sigma^2}\biggr] \\
%	
	&= \exp\biggl[-\frac{ \sum_{i=1}^n (\yi - \thetav^\top \xi)^2}{2\sigma^2}\biggr].
%	
\end{align*}
%
Moreover, note that the maximum a posteriori estimate of $\thetav,$ which is defined by
%
$$		\thetavh = \argmax_{\thetav} p(\thetav | \Xmat, \yv)	$$
%
can also be defined by 
%
%
$$		\thetavh = \argmax_{\thetav} \log \left( p(\thetav | \Xmat, \yv) \right),	$$
%
since $\log$ is a monotonically increasing function, so the maximizer is the same.
%
\begin{enumerate}
%	
  \item  If the prior distribution is  a uniform distribution over the parameter vectors $\thetav$, i.e.,
%  
	$$  q(\thetav)  \propto 1, $$
%	
	then 
%	
	\begin{eqnarray*}
		p(\thetav | \Xmat, \yv) &\propto& p(\yv | \Xmat, \thetav) q(\thetav) \\
%		
		&\propto& \exp\biggl[-\frac{ \sum_{i=1}^n (\yi - \thetav^\top \xi)^2}{2\sigma^2}\biggr].
%		
	\end{eqnarray*}
%  
	With this, 
%	
	\begin{align*}
%				
		\thetavh 
%		
		&= \argmax_{\thetav} \log \left( p(\thetav | \Xmat, \yv) \right) \\
%		
		&= \argmax_{\thetav} -\frac{ \sum_{i=1}^n (\yi - \thetav^\top \xi)^2}{2\sigma^2} \\
%		
		&= \argmin_{\thetav}  \frac{ \sum_{i=1}^n (\yi - \thetav^\top \xi)^2}{2\sigma^2} \\
%		
		&= \argmin_{\thetav}  \sum_{i=1}^n (\yi - \thetav^\top \xi)^2, \tag{$2\sigma^2$ is just a constant scaling}
%		
	\end{align*}
%
	so the  maximum a posteriori estimate coincides with the empirical risk minimizer for the L2-loss (over the linear models).
%
  \item If we choose a Gaussian distribution over the parameter vectors $\thetav$ as the prior belief, i.e.,
  %  
  $$  q(\thetav)  \propto  \exp\biggl[ -\frac{1}{2\tau^2}\thetav^\top\thetav  \biggr], \qquad \tau>0, $$
  %  
  	then 
%  
  \begin{eqnarray*}
%  	
  	p(\thetav | \Xmat, \yv) &\propto& p(\yv | \Xmat, \thetav) q(\thetav) \\
  	%		
  	&\propto& \exp\biggl[-\frac{ \sum_{i=1}^n (\yi - \thetav^\top \xi)^2}{2\sigma^2} -\frac{1}{2\tau^2}\thetav^\top\thetav \biggr] \\
%  	
	&=& \exp\biggl[-\frac{ \sum_{i=1}^n (\yi - \thetav^\top \xi)^2}{2\sigma^2} -\frac{\|\thetav\|_2^2}{2\tau^2}  \biggr]
  	%		
  \end{eqnarray*}
  %  
  With this, 
  %	
  \begin{align*}
  	%				
  	\thetavh 
  	%		
  	&= \argmax_{\thetav} \log \left( p(\thetav | \Xmat, \yv) \right) \\
  	%		
  	&= \argmax_{\thetav} -\frac{ \sum_{i=1}^n (\yi - \thetav^\top \xi)^2}{2\sigma^2} -\frac{\|\thetav\|_2^2}{2\tau^2}   \\
  	%		
  	&= \argmin_{\thetav}  \frac{ \sum_{i=1}^n (\yi - \thetav^\top \xi)^2}{2\sigma^2} + \frac{\|\thetav\|_2^2}{2\tau^2}   \\
%  	
  	&= \argmin_{\thetav}   \sum_{i=1}^n (\yi - \thetav^\top \xi)^2  + \frac{\sigma^2}{ \tau^2}  \|\thetav\|_2^2   ,
  	%		
  \end{align*}
  %
  so the  maximum a posteriori estimate coincides for the choice of $\lambda = \frac{\sigma^2}{ \tau^2}>0$ with the regularized empirical risk minimizer for the L2-loss with L2 penalty (over the linear models), i.e., the Ridge regression.
%  
  \item If we choose a Laplace distribution over the parameter vectors $\thetav$ as the prior belief, i.e.,
%  
	$$  q(\thetav)  \propto  \exp\biggl[-\frac{\sum_{i=1}^p |\thetav_i|}{\tau} \biggr], \qquad \tau>0, $$
	%  
	  	then 
%	
	\begin{eqnarray*}
		%  	
		p(\thetav | \Xmat, \yv) &\propto& p(\yv | \Xmat, \thetav) q(\thetav) \\
		%		
		&\propto& \exp\biggl[-\frac{ \sum_{i=1}^n (\yi - \thetav^\top \xi)^2}{2\sigma^2} -\frac{\sum_{i=1}^p |\thetav_i|}{\tau}  \biggr] \\
		%  	
		&=& \exp\biggl[-\frac{ \sum_{i=1}^n (\yi - \thetav^\top \xi)^2}{2\sigma^2} -\frac{\|\thetav\|_1}{\tau}  \biggr]
		%		
	\end{eqnarray*}
	%  
	With this, 
	%	
	\begin{align*}
		%				
		\thetavh 
		%		
		&= \argmax_{\thetav} \log \left( p(\thetav | \Xmat, \yv) \right) \\
		%		
		&= \argmax_{\thetav} -\frac{ \sum_{i=1}^n (\yi - \thetav^\top \xi)^2}{2\sigma^2} -\frac{\|\thetav\|_1}{\tau}   \\
		%		
		&= \argmin_{\thetav}  \frac{ \sum_{i=1}^n (\yi - \thetav^\top \xi)^2}{2\sigma^2} + \frac{\|\thetav\|_1}{\tau}  \\
		%  	
		&= \argmin_{\thetav}   \sum_{i=1}^n (\yi - \thetav^\top \xi)^2  + \frac{2\sigma^2}{ \tau}  \|\thetav\|_1   ,
		%		
	\end{align*}
	%
	so the  maximum a posteriori estimate coincides for the specific choice of $\lambda = \frac{2\sigma^2}{ \tau}$ with the regularized empirical risk minimizer for the L2-loss with L1 penalty (over the linear models), i.e., the Lasso regression.
%	
\end{enumerate}
%
