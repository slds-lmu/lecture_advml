\documentclass[a4paper]{article}
% maxwidth is the original width if it is less than linewidth
% otherwise use linewidth (to make sure the graphics do not exceed the margin)
\makeatletter
\def\maxwidth{ %
  \ifdim\Gin@nat@width>\linewidth
    \linewidth
  \else
    \Gin@nat@width
  \fi
}
\makeatother

\usepackage[utf8]{inputenc}
\usepackage{a4wide,paralist}
\usepackage{amsmath, amssymb, xfrac, amsthm}
\usepackage{dsfont}
\usepackage{xcolor}
\usepackage{amsfonts}
\usepackage{graphicx}
\usepackage{hyperref}
\usepackage{bm}

% for showing python code blocks
\usepackage{listings}
\lstdefinestyle{pythonstyle}{
    language=Python,
    backgroundcolor=\color[rgb]{0.98,0.98,0.98},
    commentstyle=\color[rgb]{0.13,0.54,0.13},
    keywordstyle=\color[rgb]{0.2,0.29,0.37},
    numberstyle=\tiny\color{gray},
    stringstyle=\color[rgb]{0.73,0.13,0.13},
    basicstyle=\ttfamily\small,
    breakatwhitespace=false,
    breaklines=true,
    captionpos=b,
    keepspaces=true,
    showspaces=false,
    showstringspaces=false,
    showtabs=false,
    tabsize=4
}
\lstset{style=pythonstyle}


% %exercise numbering
\renewcommand{\theenumi}{(\alph{enumi})}
\renewcommand{\theenumii}{\roman{enumii}}
\renewcommand\labelenumi{\theenumi}


\font \sfbold=cmssbx10

\setlength{\oddsidemargin}{0cm} \setlength{\textwidth}{16cm}


\sloppy
\parindent0em
\parskip0.5em
\topmargin-2.3 cm
\textheight25cm
\textwidth17.5cm
\oddsidemargin-0.8cm
\pagestyle{empty}


\newcommand{\kopfaml}[2]{
\hrule
\vspace{.15cm}
\begin{minipage}{\textwidth}
%akwardly i had to put \" here to make it compile correctly
	{\sf\bf Advanced Machine Learning \hfill Exercise sheet #1\\
	 \url{https://slds-lmu.github.io/i2ml/} \hfill #2}
\end{minipage}
\vspace{.05cm}
\hrule
\vspace{1cm}}


\newenvironment{allgemein}
	{\noindent}{\vspace{1cm}}

\newcounter{aufg}
\newenvironment{aufgabe}[1]
	{\refstepcounter{aufg}\textbf{Exercise \arabic{aufg}: #1}\\ \noindent}
	{\vspace{0.5cm}}

\newcounter{loes}
\newenvironment{loesung}[1]
	{\refstepcounter{loes}\textbf{Solution \arabic{loes}: #1}\\\noindent}
	{\bigskip}

\newenvironment{bonusaufgabe}
	{\refstepcounter{aufg}\textbf{Exercise \arabic{aufg}*\footnote{This
	is a bonus exercise.}:}\\ \noindent}
	{\vspace{0.5cm}}

\newenvironment{bonusloesung}
	{\refstepcounter{loes}\textbf{Solution \arabic{loes}*:}\\\noindent}
	{\bigskip}



\begin{document}

\lstset{style=rstyle}


\input{../../latex-math/basic-math.tex}
\input{../../latex-math/basic-ml.tex}

\kopfaml{11}{Multitarget Learning}

\newcommand{\ba}{\mathbf{a}}

\loesung{Multivariate Regression}{

Consider the multivariate regression setting on $\Xspace \subset \R^p$ without target features, i.e., $\Yspace=\R$ and $\mathcal{T} = \{1,\ldots,m\}.$	
Furthermore, consider the approach of learning a (simple) linear model $f_j(\xv) = \ba_j^\top \xv$ for each target $j$ independently.
For this purpose, we would face the following optimization problem:
\begin{equation*}	\label{eq:multiridge}
	\min_A \|Y - \Xmat A \|^2_F,		
\end{equation*}
where  $ \| B \|^2_F  = \sqrt{ \sum_{i=1}^n \sum_{j=1}^m B_{i,j}^2 } $ is the Frobenius norm for a matrix $B \in \R^{n \times m}$ and 
%		
\begin{equation*} \label{eq:notation}
    \Xmat = \begin{bmatrix} (\xv^{(1)} )^\top \\ \vdots \\ (\xv^{(n)})^\top \end{bmatrix}, \qquad A = [\ba_1 \quad \cdots \quad \ba_m] \,, \qquad Y = \begin{bmatrix} \yv^{(1)}\\ \vdots \\ \yv^{(n)} \end{bmatrix}.
\end{equation*}

\begin{enumerate}
	\item Show that $\hat A = (\Xmat^\top \Xmat)^{-1} \Xmat^\top Y$ is the optimal solution in this case (provided that $\Xmat^\top \Xmat$ is invertible).
	
	\textbf{Solution:}
	
	Note that for a matrix $B = (\bm{b}_1 \ldots \bm{b}_m) \in \R^{n\times m},$ it holds that
	\begin{align*} 
		\| B \|^2_F   
	 	=  \sum_{i=1}^n \sum_{j=1}^m B_{i,j}^2 
		=  \sum_{j=1}^m \sum_{i=1}^n B_{i,j}^2 
		&=  \sum_{j=1}^m \bm{b}_j^\top \bm{b}_j. 
	\end{align*}
	Thus, the function $f(A)=\|Y - \Xmat A \|^2_F$ we want to minimize can be written as
	\begin{align*}
		\|Y - \Xmat A \|^2_F 
		&= \sum_{j=1}^m (\bm{y}_j -\Xmat \bm{a}_j)^\top(\bm{y}_j -\Xmat \bm{a}_j) \\
		&= \sum_{j=1}^m \bm{y}_j^\top\bm{y}_j - 2 \bm{y}_j^\top \Xmat \bm{a}_j + \bm{a}_j^\top \Xmat^\top  \Xmat \bm{a}_j,
	\end{align*}
	where $\bm{y}_j$ is the $j$-th column of $Y.$
	Therefore, we can write $f(A) = \sum_{j=1}^m f_j(\ba_j),$ where 
	$$	f_j(\ba) = 	\bm{y}_j^\top\bm{y}_j - 2 \bm{y}_j^\top \Xmat \bm{a} + \bm{a}^\top \Xmat^\top  \Xmat \bm{a}	$$
	and we can minimize each $f_j$ separately (w.r.t.\ to $\ba_j$).
	We compute the gradient of $f_j$ and set it to $\bm{0}$ and solve w.r.t.\ $\ba:$
	\begin{align*}
		\nabla f_j(\ba) 
		&= - 2 \bm{y}_j^\top \Xmat  + 2 \Xmat^\top  \Xmat \bm{a} \stackrel{!}{=} \bm{0} \\
		&\Leftrightarrow \ba = (\Xmat^\top \Xmat)^{-1} \Xmat^\top \bm{y}_j.
	\end{align*}
	We check that this is indeed a minimum, by checking that the Hessian matrix is positive semi-definite:
	The Hessian matrix is
    \begin{align*}	
        \nabla \nabla^\top f_j(\ba) 		
        &=   2 \Xmat^\top  \Xmat.		
	\end{align*}
	It is positive semi-definite, since for any $\bm{z}\in \R^p$ it holds that
	\begin{align*}
		\bm{z}^\top (2 \Xmat^\top  \Xmat) \bm{z} 
		= 2 \bm{z}^\top  \Xmat^\top  \Xmat  \bm{z} 
		= 	2   (\Xmat  \bm{z})^\top \Xmat  \bm{z} 
		= 2 \| \Xmat  \bm{z} \|_2^2 \geq 0.
	\end{align*}
	Consequently, the minimizer of $f_j$ is $\hat{\ba}_j= (\Xmat^\top \Xmat)^{-1} \Xmat^\top \bm{y}_j,$ so that the minimizer of $f$ is 
 
	$$	\hat{A} = (\hat{\ba}_1 ~ \ldots ~ \hat{\ba}_m) = \left(	(\Xmat^\top \Xmat)^{-1} \Xmat^\top \bm{y}_1 ~ \ldots ~ (\Xmat^\top \Xmat)^{-1} \Xmat^\top \bm{y}_m		\right) = (\Xmat^\top \Xmat)^{-1} \Xmat^\top Y.	$$
	In particular, the gradient of $f$ w.r.t.\ matrix $A=(\ba_1 ~ \ldots ~ \ba_m)$ is 
	\begin{align*}
		\nabla f(A) 	
		&= ( \nabla f_1(\ba_1) ~ \ldots ~ \nabla f_m(\ba_m) )  \\
		&= \left( - 2 \bm{y}_1^\top \Xmat  + 2 \Xmat^\top  \Xmat \bm{a}_1 \quad \ldots \quad - 2 \bm{y}_m^\top \Xmat  + 2 \Xmat^\top  \Xmat \bm{a}_m\right) \\
		&= -2Y^\top \Xmat  + 2 \Xmat^\top  \Xmat A.
	\end{align*}
	Hence, a gradient descent routine with (fixed) step size $\alpha$ for $f$ would iterate as follows:
	\begin{align*}	
		\hat A \leftarrow \hat A - 2 \alpha \left( -Y^\top \Xmat  +  \Xmat^\top  \Xmat \hat A \right).	
	\end{align*}
	\item Assume that the data $(\xi,\yv^{(i)}) \in \Xspace \times \Yspace^m$ is generated\footnote{Of course, in an iid fashion and the $\xv$'s are independent of the $\bm{\eps}$'s.} according to the following statistical model
	$$	(y_1,\ldots,y_m) =	\yv = (\xi)^\top A^* + \bm{\eps}^\top,		$$
	where $A^* \in \R^{p\times m}$ and $\bm{\eps} \sim \normal(\bm{0},\bm{\Sigma}).$
	Show that the maximum likelihood estimate for $A^*$ coincides with the estimate in (a). 
	
	\textbf{Solution:}
	
	Under the statistical model it holds that $\yv^{(i)}~|~\xi \sim \normal\left(	(\xi)^\top A^*, \bm{\Sigma}	\right),$ i.e.\footnote{Note that $\yv^{(i)} - (\xi)^\top A^* $ is a row vector.},
	\begin{align}\label{def_distr}
		p(\yv^{(i)}~|~\xi,A^*) 
		&= (2\pi)^{-m/2} |\bm{\Sigma}|^{-1/2} \, \exp\left[	-\frac12 (\yv^{(i)} - (\xi)^\top A^* )	\bm{\Sigma}^{-1} (\yv^{(i)} - (\xi)^\top A^* )^\top	\right] \notag \\
		&\propto \exp\left[	-\frac12 \left(\yv^{(i)} - (\xi)^\top A^* \right)	\bm{\Sigma}^{-1} \left(\yv^{(i)} - (\xi)^\top A^* \right)^\top	\right] .
	\end{align}
	So the log-likelihood for $A$ is
	\begin{align*}
		l(A~|~\D) 
		&= \log\left(	\prod_{i=1}^n 		p(\yv^{(i)}~|~\xi,A)	\right) \\
		&\propto \log\left(	 	\exp\left[	-\frac12 \sum_{i=1}^n \left(\yv^{(i)} - (\xi)^\top A \right)	\bm{\Sigma}^{-1} \left(\yv^{(i)} - (\xi)^\top A \right)^\top	\right]	\right) \\
		&= 	-\sum_{i=1}^n  \left(\yv^{(i)} - (\xi)^\top A \right)	\bm{\Sigma}^{-1} \left(\yv^{(i)} - (\xi)^\top A \right)^\top.		 
	\end{align*}
	So, we want to maximize the function 
	\begin{align*}
		g(A) 
		&= 	-\sum_{i=1}^n  \left(\yv^{(i)} - (\xi)^\top A \right)	\bm{\Sigma}^{-1} \left(\yv^{(i)} - (\xi)^\top A \right)^\top \\
		&= 	-\sum_{i=1}^n  \yv^{(i)} \bm{\Sigma}^{-1} (\yv^{(i)})^\top  
		- 2 (\xi)^\top A \bm{\Sigma}^{-1} (\yv^{(i)})^\top 
		+ (\xi)^\top A	\bm{\Sigma}^{-1}  A^\top \xi.	
	\end{align*}
	We compute the gradient of $g$ and set it to $\bm{0}_{p\times m}$ and solve w.r.t.\ $A:$
	\begin{align*}
		\nabla g(A) 
		&= 	\sum_{i=1}^n   2 \xi \yv^{(i)}   \bm{\Sigma}^{-1}
		- 2   \xi	(\xi)^\top  A \bm{\Sigma}^{-1}   \stackrel{!}{=} \bm{0}_{p\times m} \\
		%		
		&\Leftrightarrow 	\Xmat^\top Y  \bm{\Sigma}^{-1}
		-   \Xmat^\top \Xmat  A \bm{\Sigma}^{-1}   \stackrel{!}{=} \bm{0}_{p\times m} \\
		&\Leftrightarrow A = (\Xmat^\top \Xmat)^{-1} \Xmat^\top Y,	
	\end{align*}
	where we used for computing the gradient that 
	\begin{align*}
		&\frac{\partial \bm{z}^\top \bm{B} \tilde{\bm{z}} }{\partial  \bm{B} } = \bm{z} \tilde{\bm{z}}^\top, \qquad \forall \bm{z}\in\R^n,\tilde{\bm{z}}\in \R^{m}, \bm{B}\in\R^{n \times m}, \\
		&\frac{\partial \bm{z}^\top \bm{B} \bm{C} \bm{B}^\top \tilde{\bm{z}} }{\partial  \bm{B} } 
		= 2 \bm{z} \tilde{\bm{z}}^\top \bm{B} \bm{C}^\top  , \qquad \forall \bm{z}\in\R^n,\tilde{\bm{z}}\in \R^{m}, \bm{B}\in\R^{n \times m}, \bm{C} \in \R^{n \times n}.
	\end{align*}
	Moreover, any matrix product $\Xmat^\top Y$ can be written as the sum of outer products of the column and row vectors: $ \sum_{i=1}^n    \xi \yv^{(i)}.$
	Thus, the MLE coincides with the OLS in (a).		
	\item Write a function implementing a gradient descent routine for the optimization of this linear model.
	 \item Run a small simulation study by creating 20 data sets as indicated below and test different step sizes $\alpha$ (fixed across iterations) against each other and against the state-of-the-art routine for linear models (in R, using the function \texttt{lm}, in Python, e.g., \texttt{sklearn.linear\_model.LinearRegression}).
  \begin{itemize}
    \item Compare the difference in the estimated parameter matrices $\hat A$ using the mean squared error, i.e., 
    $$ \frac{1}{m\cdot p} \sum_{i=1}^p \sum_{j=1}^m (\ba^{*}_{i,j}-\hat{\ba}_{i,j})^2$$ 
    and summarize the difference over all 100 simulation repetitions.
    \item Compare the estimation also with the James-Stein estimate of $A^*$, which is given by 
	$$  A^{JS} = \left(  \ba_1^{JS} \ldots \ba_m^{JS}  \right),$$
	where $$  \ba_j^{JS} = \left (1 - \frac{(m-2)\sigma^2}{n\|\hat{\ba}_j - \ba_j^* \|^2_2} \right )  (\hat{\ba}_j - \ba_j^*) + \ba_j^*, \quad j=1,\ldots,m.$$
	and $\hat{\ba}_j$ is the MLE for the $j$th target parameter.
    \end{itemize}

\end{enumerate}

}


\end{document}