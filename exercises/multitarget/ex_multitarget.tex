\documentclass[a4paper]{article}
\usepackage[]{graphicx}\usepackage[]{xcolor}
% maxwidth is the original width if it is less than linewidth
% otherwise use linewidth (to make sure the graphics do not exceed the margin)
\makeatletter
\def\maxwidth{ %
  \ifdim\Gin@nat@width>\linewidth
    \linewidth
  \else
    \Gin@nat@width
  \fi
}
\makeatother

\definecolor{fgcolor}{rgb}{0.345, 0.345, 0.345}
\newcommand{\hlnum}[1]{\textcolor[rgb]{0.686,0.059,0.569}{#1}}%
\newcommand{\hlstr}[1]{\textcolor[rgb]{0.192,0.494,0.8}{#1}}%
\newcommand{\hlcom}[1]{\textcolor[rgb]{0.678,0.584,0.686}{\textit{#1}}}%
\newcommand{\hlopt}[1]{\textcolor[rgb]{0,0,0}{#1}}%
\newcommand{\hlstd}[1]{\textcolor[rgb]{0.345,0.345,0.345}{#1}}%
\newcommand{\hlkwa}[1]{\textcolor[rgb]{0.161,0.373,0.58}{\textbf{#1}}}%
\newcommand{\hlkwb}[1]{\textcolor[rgb]{0.69,0.353,0.396}{#1}}%
\newcommand{\hlkwc}[1]{\textcolor[rgb]{0.333,0.667,0.333}{#1}}%
\newcommand{\hlkwd}[1]{\textcolor[rgb]{0.737,0.353,0.396}{\textbf{#1}}}%
\let\hlipl\hlkwb

\usepackage{framed}
\makeatletter
\newenvironment{kframe}{%
 \def\at@end@of@kframe{}%
 \ifinner\ifhmode%
  \def\at@end@of@kframe{\end{minipage}}%
  \begin{minipage}{\columnwidth}%
 \fi\fi%
 \def\FrameCommand##1{\hskip\@totalleftmargin \hskip-\fboxsep
 \colorbox{shadecolor}{##1}\hskip-\fboxsep
     % There is no \\@totalrightmargin, so:
     \hskip-\linewidth \hskip-\@totalleftmargin \hskip\columnwidth}%
 \MakeFramed {\advance\hsize-\width
   \@totalleftmargin\z@ \linewidth\hsize
   \@setminipage}}%
 {\par\unskip\endMakeFramed%
 \at@end@of@kframe}
\makeatother

\definecolor{shadecolor}{rgb}{.97, .97, .97}
\definecolor{messagecolor}{rgb}{0, 0, 0}
\definecolor{warningcolor}{rgb}{1, 0, 1}
\definecolor{errorcolor}{rgb}{1, 0, 0}
\newenvironment{knitrout}{}{} % an empty environment to be redefined in TeX

\usepackage{alltt}
\newcommand{\SweaveOpts}[1]{}  % do not interfere with LaTeX
\newcommand{\SweaveInput}[1]{} % because they are not real TeX commands
\newcommand{\Sexpr}[1]{}       % will only be parsed by R




\usepackage[utf8]{inputenc}
%\usepackage[ngerman]{babel}
\usepackage{a4wide,paralist}
\usepackage{amsmath, amssymb, xfrac, amsthm}
\usepackage{dsfont}
%\usepackage[usenames,dvipsnames]{xcolor}
\usepackage{xcolor}
\usepackage{amsfonts}
\usepackage{graphicx}
\usepackage{caption}
\usepackage{subcaption}
\usepackage{framed}
\usepackage{multirow}
\usepackage{bytefield}
\usepackage{csquotes}
\usepackage[breakable, theorems, skins]{tcolorbox}
\usepackage{hyperref}
\usepackage{cancel}
\usepackage{bm}
\usepackage{algorithm}
\usepackage{algpseudocode}


\input{../../style/common}

\tcbset{enhanced}

\DeclareRobustCommand{\mybox}[2][gray!20]{%
	\iffalse
	\begin{tcolorbox}[   %% Adjust the following parameters at will.
		breakable,
		left=0pt,
		right=0pt,
		top=0pt,
		bottom=0pt,
		colback=#1,
		colframe=#1,
		width=\dimexpr\linewidth\relax,
		enlarge left by=0mm,
		boxsep=5pt,
		arc=0pt,outer arc=0pt,
		]
		#2
	\end{tcolorbox}
	\fi
}

\DeclareRobustCommand{\myboxshow}[2][gray!20]{%
%	\iffalse
	\begin{tcolorbox}[   %% Adjust the following parameters at will.
		breakable,
		left=0pt,
		right=0pt,
		top=0pt,
		bottom=0pt,
		colback=#1,
		colframe=#1,
		width=\dimexpr\linewidth\relax,
		enlarge left by=0mm,
		boxsep=5pt,
		arc=0pt,outer arc=0pt,
		]
		#2
	\end{tcolorbox}
%	\fi
}


%exercise numbering
\renewcommand{\theenumi}{(\alph{enumi})}
\renewcommand{\theenumii}{\roman{enumii}}
\renewcommand\labelenumi{\theenumi}


\font \sfbold=cmssbx10

\setlength{\oddsidemargin}{0cm} \setlength{\textwidth}{16cm}


\sloppy
\parindent0em
\parskip0.5em
\topmargin-2.3 cm
\textheight25cm
\textwidth17.5cm
\oddsidemargin-0.8cm
\pagestyle{empty}

\newcommand{\kopf}[2]{
\hrule
\vspace{.15cm}
\begin{minipage}{\textwidth}
%akwardly i had to put \" here to make it compile correctly
	{\sf\bf Introduction to Machine Learning \hfill Exercise sheet #1\\
	 \url{https://slds-lmu.github.io/i2ml/} \hfill #2}
\end{minipage}
\vspace{.05cm}
\hrule
\vspace{1cm}}

\newcommand{\kopfic}[2]{
\hrule
\vspace{.15cm}
\begin{minipage}{\textwidth}
%akwardly i had to put \" here to make it compile correctly
	{\sf\bf Introduction to Machine Learning \hfill Live Session #1\\
	 \url{https://slds-lmu.github.io/i2ml/} \hfill #2}
\end{minipage}
\vspace{.05cm}
\hrule
\vspace{1cm}}

\newcommand{\kopficsl}[2]{
\hrule
\vspace{.15cm}
\begin{minipage}{\textwidth}
%akwardly i had to put \" here to make it compile correctly
	{\sf\bf Supervised Learning \hfill Live Session #1\\
	 \url{https://slds-lmu.github.io/i2ml/} \hfill #2}
\end{minipage}
\vspace{.05cm}
\hrule
\vspace{1cm}}

\newcommand{\kopfaml}[2]{
\hrule
\vspace{.15cm}
\begin{minipage}{\textwidth}
%akwardly i had to put \" here to make it compile correctly
	{\sf\bf Advanced Machine Learning \hfill Exercise sheet #1\\
	 \url{https://slds-lmu.github.io/i2ml/} \hfill #2}
\end{minipage}
\vspace{.05cm}
\hrule
\vspace{1cm}}


\newcommand{\kopfdive}[1]{
\hrule
\vspace{.15cm}
\begin{minipage}{\textwidth}
%akwardly i had to put \" here to make it compile correctly
	{\sf\bf Supervised Learning \hfill Deep Dive\\
	 \url{https://slds-lmu.github.io/i2ml/} \hfill #1}
\end{minipage}
\vspace{.05cm}
\hrule
\vspace{1cm}}

\newcommand{\kopfsl}[2]{
\hrule
\vspace{.15cm}
\begin{minipage}{\textwidth}
%akwardly i had to put \" here to make it compile correctly
	{\sf\bf Supervised Learning \hfill Exercise sheet #1\\
	 \url{https://slds-lmu.github.io/i2ml/} \hfill #2}
\end{minipage}
\vspace{.05cm}
\hrule
\vspace{1cm}}

\newenvironment{allgemein}
	{\noindent}{\vspace{1cm}}

\newcounter{aufg}
\newenvironment{aufgabe}[1]
	{\refstepcounter{aufg}\textbf{Exercise \arabic{aufg}: #1}\\ \noindent}
	{\vspace{0.5cm}}

\newcounter{loes}
\newenvironment{loesung}[1]
	{\refstepcounter{loes}\textbf{Solution \arabic{loes}: #1}\\\noindent}
	{\bigskip}

\newenvironment{bonusaufgabe}
	{\refstepcounter{aufg}\textbf{Exercise \arabic{aufg}*\footnote{This
	is a bonus exercise.}:}\\ \noindent}
	{\vspace{0.5cm}}

\newenvironment{bonusloesung}
	{\refstepcounter{loes}\textbf{Solution \arabic{loes}*:}\\\noindent}
	{\bigskip}



\begin{document}
% !Rnw weave = knitr



\input{../../latex-math/basic-math.tex}
\input{../../latex-math/basic-ml.tex}

\kopfaml{9}{Multi-target Prediction}

\aufgabe{Multivariate Regression}{


\newcommand{\ba}{\mathbf{a}}

%
	%
	Consider the multivariate regression setting on $\Xspace \subset \R^p$ without target features, i.e., $\Yspace=\R$ and $\mathcal{T} = \{1,\ldots,m\}.$	
%	
	Furthermore, consider the approach of learning a (simple) linear model $f_j(\xv) = \ba_j^\top \xv$ for each target $j$ independently.
%	
	For this purpose, we would face the following optimization problem:
%	
	\begin{equation*}	\label{eq:multiridge}
%		
		\min_A \|Y - \Xmat A \|^2_F,
%		
	\end{equation*}
%
	where  $ \| B \|_F  = \sqrt{ \sum_{i=1}^n \sum_{j=1}^m B_{i,j}^2 } $ is the Frobenius norm for a matrix $B \in \R^{n \times m}$ and 
	%		
	\begin{equation*}
		\label{eq:notation}
		\Xmat = \begin{bmatrix} (\xv^{(1)} )^\top \\ \vdots \\ (\xv^{(n)})^\top \end{bmatrix}, \qquad A = [\ba_1 \quad \cdots \quad \ba_m] \,, \qquad Y = \begin{bmatrix} \yv^{(1)}\\ \vdots \\ \yv^{(n)} \end{bmatrix}.
	\end{equation*}
	
	\begin{enumerate}
%	
	\item Show that $\hat A = (\Xmat^\top \Xmat)^{-1} \Xmat^\top Y$ is the optimal solution in this case (provided that $\Xmat^\top \Xmat$ is invertible).
%	
	\item Assume that the data $(\xi,\yv^{(i)}) \in \Xspace \times \Yspace^m$ is generated\footnote{Of course, in an iid fashion and the $\xv$'s are independent of the $\bm{\eps}$'s.} according to the following statistical model
%	
	$$	(y_1,\ldots,y_m) =	\yv = (\xi)^\top A^* + \bm{\eps}^\top,		$$
%	
	where $A^* \in \R^{p\times m}$ and $\bm{\eps} \sim \normal(\bm{0},\bm{\Sigma}).$
%
	Show that the maximum likelihood estimate for $A^*$ coincides with the estimate in (a). 
	%		
	\item Write a function implementing a gradient descent routine for the optimization of this linear model. Start with (for R):
%	
\begin{knitrout}
\definecolor{shadecolor}{rgb}{0.969, 0.969, 0.969}\color{fgcolor}\begin{kframe}
\begin{alltt}
\hlcom{#' @param step_size the step_size in each iteration}
\hlcom{#' @param X the feature input matrix X}
\hlcom{#' @param Y the score matrix Y}
\hlcom{#' @param A a starting value for the parameter matrix}
\hlcom{#' @param eps a small constant measuring the changes in each update step. }
\hlcom{#' Stop the algorithm if the estimated model parameters do not change}
\hlcom{#' more than \textbackslash{}code\{eps\}.}

\hlcom{#' @return a parameter matrix A}
\hlstd{gradient_descent} \hlkwb{<-} \hlkwa{function}\hlstd{(}\hlkwc{step_size}\hlstd{,} \hlkwc{X}\hlstd{,} \hlkwc{Y}\hlstd{,} \hlkwc{A}\hlstd{,} \hlkwc{eps} \hlstd{=} \hlnum{1e-8}\hlstd{)\{}

  \hlcom{# >>> do something <<<}

  \hlkwd{return}\hlstd{(A)}

\hlstd{\}}
\end{alltt}
\end{kframe}
\end{knitrout}
%
\emph{Hint:} You have computed the gradient in (a).
%
 \item Run a small simulation study by creating 20 data sets as indicated below and test different step sizes $\alpha$ (fixed across iterations) against each other and against the state-of-the-art routine for linear models (in R, using the function \texttt{lm}, in Python, e.g., \texttt{sklearn.linear\_model.LinearRegression}).
 %
  \begin{itemize}
%  	
    \item Compare the difference in the estimated parameter matrices $\hat A$ using the mean squared error, i.e., 
    %
    $$ \frac{1}{m\cdot p} \sum_{i=1}^p \sum_{j=1}^m (\ba^{*}_{i,j}-\hat{\ba}_{i,j})^2$$ 
    %
    and summarize the difference over all 100 simulation repetitions.
%    
    \item Compare the estimation also with the James-Stein estimate of $A^*$, which is given by 
%    
	$$  A^{JS} = \left(  \ba_1^{JS} \ldots \ba_m^{JS}  \right),$$
%	
	where $$  \ba_j^{JS} = \left (1 - \frac{(m-2)\sigma^2}{n\|\hat{\ba}_j - \ba_j^* \|^2_2} \right )  (\hat{\ba}_j - \ba_j^*) + \ba_j^*, \quad j=1,\ldots,m.$$
%	
	and $\hat{\ba}_j$ is the MLE for the $j$th target parameter.
%    
    \end{itemize}
\begin{knitrout}
\definecolor{shadecolor}{rgb}{0.969, 0.969, 0.969}\color{fgcolor}\begin{kframe}
\begin{alltt}
\hlcom{# settings}
\hlstd{n} \hlkwb{<-} \hlnum{10000}
\hlstd{p} \hlkwb{<-} \hlnum{100}
\hlstd{m} \hlkwb{<-} \hlnum{6}
\hlstd{nr_sims} \hlkwb{<-} \hlnum{20}

\hlcom{# create data (only once)}
\hlstd{X} \hlkwb{<-} \hlkwd{matrix}\hlstd{(}\hlkwd{rnorm}\hlstd{(n}\hlopt{*}\hlstd{p),} \hlkwc{ncol}\hlstd{=p)}
\hlstd{A_truth} \hlkwb{<-} \hlkwd{matrix}\hlstd{(}\hlkwd{runif}\hlstd{(p}\hlopt{*}\hlstd{m,} \hlopt{-}\hlnum{2}\hlstd{,} \hlnum{2}\hlstd{),}\hlkwc{ncol}\hlstd{=m)}
\hlstd{f_truth} \hlkwb{<-} \hlstd{X}\hlopt\hlstd{A_truth}

\hlcom{# create result object}
\hlstd{result_list} \hlkwb{<-} \hlkwd{vector}\hlstd{(}\hlstr{"list"}\hlstd{, nr_sims)}

\hlkwa{for}\hlstd{(sim_nr} \hlkwa{in} \hlstd{nr_sims)}
\hlstd{\{}

  \hlcom{# create response}
  \hlstd{Y} \hlkwb{<-} \hlstd{f_truth} \hlopt{+} \hlkwd{rnorm}\hlstd{(n}\hlopt{*}\hlstd{m,} \hlkwc{sd} \hlstd{=} \hlnum{2}\hlstd{)}

  \hlcom{# >>> do something <<<}


  \hlcom{# save results in list (performance, time)}
  \hlstd{result_list[[sim_nr]]} \hlkwb{<-} \hlstd{add_something_meaningful_here}

\hlstd{\}}
\end{alltt}
\end{kframe}
\end{knitrout}
%

\end{enumerate}
} 

\aufgabe{Conditional Random Fields vs. Structured SVMs}{

%
Similar to probabilistic classifier chains, conditional random fields try to model the conditional distribution $\P(\yv~|~\xv)$ by means of
%
$$		\pi(\xv,\yv) = \frac{\exp(s(\xv,\yv))}{\sum_{\yv' \in \Yspace^m} \exp(s(\xv,\yv'))},	$$
%
where $x\in \Xspace$ and $\yv \in \Yspace$ with $\Yspace$ being a finite set (e.g., multi-label classification), and $s:\Xspace \times \Yspace \to \R$ being a scoring function.
%
Training of a conditional random field is based on (regularized) empirical risk minimization using the negative log-loss:
%
$$ \ell_{log}(\xv,\yv,s) = \log\left(\sum_{\yv' \in \Yspace^m} \exp(s(\xv,\yv'))\right) - s(\xv,\yv).	$$
%
Predictions are then made by means of
%
\begin{align}\label{prediction}
%	
	h(\xv) = \argmax_{\yv \in \Yspace^m} s(\xv,\yv).
%	
\end{align}
%
Structured Support Vector Machines (Structured SVMs) are also using scoring functions for the prediction, but use the structured hinge loss for the (regularized) empirical risk minimization approach:
%
$$	\ell_{shinge}(\xv,\yv,s) = \max_{\yv' \in \Yspace^m} \left( \ell(\yv,\yv') + s(\xv,\yv')	 - s(\xv,\yv) \right),	$$
%
where $\ell:\Yspace^m \times \Yspace^m \to \R$ is some target loss function (e.g., Hamming loss or subset 0/1 loss).

Show that if we use scoring functions $s$ of the form
%
$$	s(\xv,\yv) = \sum_{j=1}^m s_j(\xv,y_j),	$$
%
where $s_j:\Xspace \times \Yspace \to \R$ are scoring functions for the $j$-th target, then
%
\begin{enumerate}
%	
	\item conditional random fields are very well suited to model the case, where the distributions of the targets $y_1,\ldots,y_m$ are conditionally independent,
%	
	\item the structured hinge loss corresponds to the multiclass hinge loss for the targets if we use the (non-averaged) Hamming loss for $\ell(\yv,\yv')= \sum_{j=1}^m \mathds{1}_{[y_j \neq y_j']} $, i.e.,
%	
	$$	\ell_{shinge}(\xv,\yv,s) =  \sum_{j=1}^m \max_{y_j' \in \Yspace} \left( \mathds{1}_{[y_j \neq y_j']} + s_j(\xv,y_j')	- s_j(\xv,y_j) \right).	$$
%	
\end{enumerate} 
}


\end{document}
